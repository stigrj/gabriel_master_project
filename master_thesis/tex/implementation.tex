\makeatletter
\def\input@path{{../}}
\makeatother
\documentclass[../master_thesis.tex]{subfiles}
\begin{document}
\chapter{Implementation}\label{chap:implementation}
\section{Fosso-Tande \& Harrison's \ac{GPE}}
Fosso-Tande and Harrison described in \cite{FossoTande:2013ka} a \ac{GPE} that
included a dielectric function which was analytical through the cavity boundary.
This model was explained with more detail in chapter \ref{Solvent_effect} and
it will not be explained again, here we will show how it was implemented on
MRchem.
\subsection{Cavity Function}
The first step was to create a cavity function as in \cite{FossoTande:2013ka}.
This was done by creating a cavity object that stored the coordinates of nuclei
$\rvec_I$ and their characteristic radii $R_I$. When a point $\rvec$ is evaluated
in the Cavity function it will return $0$ if it is outside the cavity and $1$ if
it is inside the Interlocking spheres cavity defined by the nuclei coordinates
and their radii. Additionally we wanted to be able to change the width of the
boundary $\sigma$. The structure of the cavity object is as follows

\begin{algorithm}
  \caption{Cavity object}
  \begin{algorithmic}
    \STATE \underline{\textbf{Initialize Cavity:}}
    \STATE \textbf{Input :} Coordinates, Radii, Width
    \STATE $\sigma = $ Width
    \FOR{ All Coordinates$_I$ and Radii$_I$}
     \STATE$\rvec_I = $ Coordinates$_I$
     \STATE$R_I = $ Radii
    \ENDFOR
    \STATE
    \STATE \underline{\textbf{Evaluate Cavity:}}
    \STATE \textbf{Input: } $\rvec$
    \FOR{All $\rvec_I$}
      \STATE $s_I = \abs{\rvec - \revc_I} - R_I$
      \STATE $\Theta_I$
    \ENDFOR
  \end{algorithmic}
\end{algorithm}


\subsection{Dielectric Function}
\subsection{Reaction Potential}
\subsection{The iterative method}
\section{Variational implementation}
\section{Minor adjustments and advantages}

\section{Software used}
The problem was first implemented in Vampyr %cite and get the right font
which is a python %
Then it was implemented in C++ using MRchem %cite and get the right font
Both implementations are identical, except for slight changes for
performance improvement, such as a KAIN accelerator, %cite and correct font
iterating through $V_r$ instead of $V_{tot}$ % more explanation on this in previous or this section
and keeping the potential from cycle to cycle
in the c++ version of the model.


\begin{acronym}
\acro{AUS}[\href{https://www.sigma2.no/content/advanced-user-support}{AUS}]{Numerical Methods in Quantum Chemistry}
\acro{BO}{Born-Oppenheimer}
\acro{CTCC}[\href{http://www.ctcc.no}{CTCC}]{Centre for Theoretical and Computational Chemistry}
\acro{DC}{Dielectric Continuum}
\acro{DFT}{Density Functional Theory}
\acro{EFP}{Effective Fragment Potential}
\acro{EU}{European Union}
\acro{HF}{Hartree-Fock}
\acro{Hylleraas}[\href{https://www.mn.uio.no/hylleraas/english/}{Hylleraas}]{Hylleraas
  Centre for Quantum Molecular Sciences}
\acro{HPC}{High Performance Computing}
\acro{KTH}{Royal Institute of Technology}
\acro{LDA}{Local Density Approximation}
\acro{MCD}{Magnetic Circular Dichroism}
\acro{MCSCF}{Multiconfiguration Self Consistent Field}
\acro{MM}{Molecular Mechanics}
\acro{MW}{Multiwavelet}
\acro{NFR}{Norwegian Research Council}
\acro{NMQC}[\href{http://www.ctcc.no/events/conferences/2015/numeric-conference/}{NMQC}]{Numerical Methods in Quantum Chemistry}
\acro{NOTUR}[\href{https://www.notur.no/}{NOTUR}]{Norwegian Metacenter for Computational Science}
\acro{PCM}{Polarizable Continuum Model}
\acro{PI}{Primcipal Investigator}
\acro{QC}{Quantum Chemistry}
\acro{QM}{Quantum Mechanics}
\acro{QM/MM}{Quantum Mechanics/Molecular Mechanics}
\acro{ROA}{Raman Optical Activity}
\acro{SC}{semiconductor}
\acro{SCF}{Self Consistent Field}
\acro{SHG}{Second Harmonic Genertation}
\acro{STSM}{Short-term scientific mission}
\acro{TPA}{Two-Photon Absorption}
\acro{WP}{Work Package}
\acro{CBS}{Complete Basis Set}
\acro{TCG}{Theoretical Chemistry Group}
\acro{vdW}{van der Waals}
\acro{SE}{Schrödinger Equation}
\acro{PES}{Potential Energy Surface}
\acro{LCAO}{Linear Combination of Atomic Orbitals}
\acro{MRA}{Multi-Resolution Analysis}
\acro{NS}{Nonstandard}
\acro{GPE}{Generalized Poisson Equation}
\acro{COSMO}{Conductor-like Screening model}
\acro{IEF}{Integral equation formalism}
\acro{SVPE}{Surface and Volume Polarization for Electrostatic}
\acro{ASC}{Apparent Surface Charge}
\acro{SCRF}{Self Consistent Reaction Field}
\acro{STO}{Slater-type orbitals}
\end{acronym}

\biblio
\end{document}
