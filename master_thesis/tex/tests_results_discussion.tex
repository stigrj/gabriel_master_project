\makeatletter
\def\input@path{{../}}
\makeatother
\documentclass[../master_thesis.tex]{subfiles}
\begin{document}
\chapter{Results}
\section{Description of Systems}
In order to test our implementation of the solvent effect as done in \cite{FossoTande:2013ka}
we will be comparing results with two sets of data, calculated data from Gaussian
computational chemistry software and test data from Chipman \cite{Chipman2002}.
The solute molecules in question are \ce{H_2O}, \ce{NO^+}, \ce{CN^-},
\ce{CH_3CONH_2} and \ce{Li^+}. The solvent is water with a dielectric constant
$\epsinf$ of $78.304$.

The first tests consisted of using only a single cavity for all the molecule and
increasing its radius from $3.6 a.u.$ to $5.0 a.u.$ by increments of $0.1 a.u.$.
This was done to all the solute molecules above, except for  \ce{CH_3CONH_2} which
was too large for these tests. This was done in Mrchem with relative precision of
$1.0e-5$ and a cavity boundary width of $\sigma = 0.2$. In Gaussian these tests were
ran using \ac{HF} method with coupled cluster basis sets (keyword cc-pVXZ where X
can be replaced by D, T, Q and 5) and increasing their completeness. These were
began as normal coupled cluster, then augmented and finally double augmented.

Then we tested atom centered intersecting spheres cavity (Here called \ac{ABC})
with radii equal to the Bondi radii of the atoms in the molecules \cite{doi:10.1021/j100785a001} scaled by $1.2$
\cite{Tomasi:1994wt}, which was done for all solutes (\ce{Li^+} was not necessary as it
is just a single atom). The \ac{ABC} test was performed a second time, increasing the radii by $0.2 a.u.$.
Both sets were ran in Mrchem and Gaussian, with the same specifications as mentioned above.

For the second sets of tests we had to compare our results with \cite{Chipman2002}.
Since Chipman used isodensity cavities it was not as simple to reproduce their results.
In order to be able to compare my results with theirs I decided to compare all my
calculations across all radii and see what trend they had.

The reasoning behind using several radii in the tests is because of the amount of
charge that escapes the cavity. Since the cavity function we are using is analytical
instead of a boundary condition, we can't assume all the charge distribution
is contained within it. By varying the radii we can sample what radii in Mrchem
is best suited to represent radii in calculations in Gaussian or in \cite{Chipman2002}.

Lithium was ran with three different relative precision in Mrchem. This was done
in order to see how the precision affected the difference between the the References
(Gaussian and Chipman) and the calculated values.

The best outcome is if the results from Mrchem differ from the reference (Gaussian and Chipman)
by an additive coonstant, meaning that our results are equivalent to the reference.
IF the results at first do not have this constant difference, then it is hoped that
there is an assymptote at which the mrchem results behave almost parallel to
the reference ones.

Higher precision in the Lithium tests is expected to improve the values obtained
with respect to the references.

\section{Data}
\subsection{\ac{ABC} tests}
The data tables containing all results can be found at the end of the chapter in section \ref{Datatables}, following
tables will show a small sample so the reader can make better understand the tables
in the appendixes.

The following table \ref{tab:rawwaterdata}  presents the data for the energy
calculations of \ce{H_2O} with the three first cavity radii as an example.
These are the total \ac{SCF} energy values.
The same type of tables were used for the rest of the
systems.
\begin{table}[htbp]
\caption{Total Energy Calculations example for Water in Water. Energy in Hartree and radii of the cavity in Bohr}
\begin{center}
\begin{tabular}{|l|r|r|r|r|}
\hline
Basis & $R =3.6$ & $R=3.7$ & $R=3.8$ & $\ldots$\\  \hline
Cc-pVDZ & -7.6039e+01 & -7.6038e+01 & -7.6036e+01 & $\ldots$\\ \hline
Cc-pVTZ & -7.6070e+01 & -7.6069e+01 & -7.6067e+01 & $\ldots$\\ \hline
Cc-pVQZ & -7.6078e+01 & -7.6076e+01 & -7.6075e+01 & $\ldots$\\ \hline
Cc-pV5Z & -7.6080e+01 & -7.6079e+01 & -7.6077e+01 & $\ldots$\\ \hline
Aug-cc-pVDZ & -7.6054e+01 & -7.6053e+01 & -7.6052e+01 & $\ldots$\\ \hline
Aug-cc-pVTZ & -7.6074e+01 & -7.6072e+01 & -7.6071e+01 & $\ldots$\\ \hline
Aug-cc-pVQZ & -7.6079e+01 & -7.6077e+01 & -7.6076e+01 & $\ldots$\\ \hline
Aug-cc-pV5Z & -7.6080e+01 & -7.6079e+01 & -7.6077e+01 & $\ldots$\\ \hline
daug-cc-pVDZ & -7.6055e+01 & -7.6053e+01 & -7.6052e+01 & $\ldots$\\ \hline
daug-cc-pVTZ & -7.6074e+01 & -7.6072e+01 & -7.6071e+01 & $\ldots$\\ \hline
daug-cc-pVQZ & -7.6079e+01 & -7.6077e+01 & -7.6076e+01 & $\ldots$\\ \hline
daug-cc-pV5Z & -7.6080e+01 & -7.6079e+01 & -7.6077e+01 & $\ldots$\\ \hline
mrchem & -7.6085E+01 & -7.6083E+01 & -7.6081E+01 & $\ldots$\\ \hline
\end{tabular}
\end{center}
\label{tab:rawwaterdata}
\end{table}

In order to get the Reaction field
energies $E_r$ one needs the gas state Energy $E_{vac}$ for a given
basis set and subtract it from the total solvent energy $E_{tot}$ calculated for
that basis set.
\begin{equation}
  E_r = E_{tot} - E_{vac}
\end{equation}
Examples for the first three radii for the results of the above operation can be
seen in table \ref{tab:Erwatdata}

\begin{table}[htbp]
\caption{Reaction Field Energy Calculations example for Water in Water. Energy in Hartree and radii of the cavity in Bohr}
\begin{center}
\begin{tabular}{|l|r|r|r|r|}
\hline
basis & 3.6 & 3.7 & 3.8 & $\ldots$\\ \hline
Cc-pVDZ & -1.2450E-02 & -1.0998E-02 & -9.7804E-03 & $\ldots$\\ \hline
Cc-pVTZ & -1.3097E-02 & -1.1545E-02 & -1.0243E-02 & $\ldots$\\ \hline
Cc-pVQZ & -1.3218E-02 & -1.1651E-02 & -1.0334E-02 & $\ldots$\\ \hline
Cc-pV5Z & -1.3284E-02 & -1.1713E-02 & -1.0393E-02 & $\ldots$\\ \hline
Aug-cc-pVDZ & -1.3190E-02 & -1.1634E-02 & -1.0328E-02 & $\ldots$\\ \hline
Aug-cc-pVTZ & -1.3238E-02 & -1.1670E-02 & -1.0353E-02 & $\ldots$\\ \hline
Aug-cc-pVQZ & -1.3221E-02 & -1.1655E-02 & -1.0338E-02 & $\ldots$\\ \hline
Aug-cc-pV5Z & -1.3223E-02 & -1.1655E-02 & -1.0337E-02 & $\ldots$\\ \hline
daug-cc-pVDZ & -1.3228E-02 & -1.1665E-02 & -1.0351E-02 & $\ldots$\\ \hline
daug-cc-pVTZ & -1.3243E-02 & -1.1675E-02 & -1.0357E-02 & $\ldots$\\ \hline
daug-cc-pVQZ & -1.3223E-02 & -1.1656E-02 & -1.0340E-02 & $\ldots$\\ \hline
daug-cc-pV5Z & -1.3224E-02 & -1.1655E-02 & -1.0337E-02 & $\ldots$\\ \hline
mrchem & -1.8036E-02 & -1.5494E-02 & -1.3437E-02 & $\ldots$\\ \hline
\end{tabular}
\end{center}
\label{tab:Erwatdata}
\end{table}

Following are figures \ref{fig:watEnergyplots},
\ref{fig:nopEnergyplots} and \ref{fig:cyanEnergyplots}
that show the energies of the mrchem calculations plotted
against the energies from the Gaussian calculation for water, \ce{NO^+}, and
\ce{CN^-} respectively. Many of the Mrchem results for \ce{CN^-} and \ce{NO^+}
around the interval $(4.0, 4.5)$ diverged and finished or never finished
running due to divergence, due to this these results were removed from the figures so as to facilitate
the visualization of the trends across the radii. The values which deverged but still
finished running can be viewed in section \ref{Datatables}.

\begin{figure}[h!]
  \centering
  \begin{subfigure}[b]{0.75\linewidth}
    \includegraphics[width=\linewidth]{img/Erwat.png}
  \end{subfigure}
  \begin{subfigure}[b]{0.75\linewidth}
    \includegraphics[width=\linewidth]{img/Eraugwat.png}
  \end{subfigure}
  \begin{subfigure}[b]{0.75\linewidth}
    \includegraphics[width=\linewidth]{img/Erdaugwat.png}
  \end{subfigure}
  \caption{Reaction field energy of Water in a water solution, calculated with mrchem
  and with different basis sets in Gaussian}
  \label{fig:watEnergyplots}
\end{figure}

\begin{figure}[h!]
  \centering
  \begin{subfigure}[b]{\linewidth}
    \includegraphics[width=\linewidth]{img/Ernop.png}
  \end{subfigure}
  \begin{subfigure}[b]{\linewidth}
    \includegraphics[width=\linewidth]{img/Eraugnop.png}
  \end{subfigure}
  \begin{subfigure}[b]{\linewidth}
    \includegraphics[width=\linewidth]{img/Erdaugnop.png}
  \end{subfigure}
  \caption{Reaction field energy of \ce{NO^+} in a water solution, calculated with mrchem
  and with different basis sets in Gaussian}
  \label{fig:nopEnergyplots}

\end{figure}
\begin{figure}[h!]
  \centering
  \begin{subfigure}[b]{\linewidth}
    \includegraphics[width=\linewidth]{img/Ercyan.png}
  \end{subfigure}
  \begin{subfigure}[b]{\linewidth}
    \includegraphics[width=\linewidth]{img/Eraugcyan.png}
  \end{subfigure}
  \begin{subfigure}[b]{\linewidth}
    \includegraphics[width=\linewidth]{img/Erdaugcyan.png}
  \end{subfigure}
  \caption{Reaction field energy of \ce{CN^-} in a water solution, calculated with mrchem
  and with different basis sets in Gaussian}
  \label{fig:cyanEnergyplots}
\end{figure}

The Mrchem energy values $E_{Mrchem}$ for each radii were compared to the
corresponding values of each of the different basis set calculations in
Gaussian  $E_{Gaussian}^{basis}$ by finding the relative difference $d_r$
between them as
\begin{equation}\label{eq:reldiff}
  d_r = \frac{E_{Gaussian}^{basis} - E_{Mrchem}}{E_{Mrchem}}
\end{equation}
The operation above \ref{eq:reldiff} was applied to all the values for all the
substrate molecules, giving the following figures \ref{fig:watreldiff},
\ref{fig:nopreldiff} % figures coming soon , and \ref{fig:cyanreldiff}.

\begin{figure}[h!]
  \centering
  \begin{subfigure}[b]{\linewidth}
    \includegraphics[width=\linewidth]{img/watreldiff.png}
  \end{subfigure}
  \begin{subfigure}[b]{\linewidth}
    \includegraphics[width=\linewidth]{img/wataugreldiff.png}
  \end{subfigure}
  \begin{subfigure}[b]{\linewidth}
    \includegraphics[width=\linewidth]{img/watdaugreldiff.png}
  \end{subfigure}
  \caption{Relative difference between the Reaction field energy of Water in a water solution calculated with mrchem
  and with different basis sets in Gaussian}
  \label{fig:watreldiff}
\end{figure}

\begin{figure}[h!]
  \centering
  \begin{subfigure}[b]{\linewidth}
    \includegraphics[width=\linewidth]{img/nopreldiff.png}
  \end{subfigure}
  \begin{subfigure}[b]{\linewidth}
    \includegraphics[width=\linewidth]{img/nopaugreldiff.png}
  \end{subfigure}
  \begin{subfigure}[b]{\linewidth}
    \includegraphics[width=\linewidth]{img/nopdaugreldiff.png}
  \end{subfigure}
  \caption{Relative difference between the Reaction field energy of \ce{NO^+} in a water solution calculated with MrChem
  and with different basis sets in Gaussian}
  \label{fig:nopreldiff}
\end{figure}


We have observed that when, in Mrchem,  using a slightly bigger cavity than
Gaussian would give us a better difference between the Mrchem and Gaussian values.
In the following figures we compared a calculation in gaussian with a given radius $r$
against an Mrchem calculation with radius $r + 0.2$Bohr for all the tests

We ran \ac{ABC} calcualtions in the same manner as above, here are the results for
each of the substrates in tables %numerate them
As we did above we increased the radius with $0.2$ Bohr in the hopes to see a
better correspondence


\subsection{Chipman Comparison}
In this part of the tests we simply compared our calculations to Chipman \cite{Chipman2002}.
They ran several calculations using different \ac{PCM} models, some of which were
explained in section \ref{approchessolv}. Here we will simply present our results
compared to theirs.


\section{Discussion}
\section{Conclusion}
\section{Areas of improvement/future development}

\section{Data tables}\label{Datatables}
\begin{sidewaystable}[h]

  \ttabbox{
  \resizebox{\textwidth}{!}{
  \begin{tabular}{|l|r|r|r|r|r|r|r|r|r|r|r|r|r|r|r|r|}
    \hline
    basis & \multicolumn{1}{l|}{vacuum E} & 3.6 & 3.7 & 3.8 & 3.9 & 4 & 4.1 & 4.2 & 4.3 & 4.4 & 4.5 & 4.6 & 4.7 & 4.8 & 4.9 & 5 \\ \hline
    Cc-pVDZ & -7.6027E+01 & -7.6039E+01 & -7.6038E+01 & -7.6036E+01 & -7.6035E+01 & -7.6035E+01 & -7.6034E+01 & -7.6033E+01 & -7.6033E+01 & -7.6032E+01 & -7.6032E+01 & -7.6031E+01 & -7.6031E+01 & -7.6031E+01 & -7.6030E+01 & -7.6030E+01 \\ \hline
    Cc-pVTZ & -7.6057E+01 & -7.6070E+01 & -7.6069E+01 & -7.6067E+01 & -7.6066E+01 & -7.6065E+01 & -7.6064E+01 & -7.6064E+01 & -7.6063E+01 & -7.6063E+01 & -7.6062E+01 & -7.6062E+01 & -7.6061E+01 & -7.6061E+01 & -7.6061E+01 & -7.6060E+01  \\ \hline
    Cc-pVQZ & -7.6065E+01 & -7.6078E+01 & -7.6076E+01 & -7.6075E+01 & -7.6074E+01 & -7.6073E+01 & -7.6072E+01 & -7.6071E+01 & -7.6071E+01 & -7.6070E+01 & -7.6070E+01 & -7.6069E+01 & -7.6069E+01 & -7.6069E+01 & -7.6068E+01 & -7.6068E+01  \\ \hline
    Cc-pV5Z & -7.6067E+01 & -7.6080E+01 & -7.6079E+01 & -7.6077E+01 & -7.6076E+01 & -7.6075E+01 & -7.6074E+01 & -7.6074E+01 & -7.6073E+01 & -7.6073E+01 & -7.6072E+01 & -7.6072E+01 & -7.6071E+01 & -7.6071E+01 & -7.6071E+01 & -7.6070E+01  \\ \hline
    Aug-cc-pVDZ & -7.6041E+01 & -7.6054E+01 & -7.6053E+01 & -7.6052E+01 & -7.6050E+01 & -7.6050E+01 & -7.6049E+01 & -7.6048E+01 & -7.6047E+01 & -7.6047E+01 & -7.6046E+01 & -7.6046E+01 & -7.6046E+01 & -7.6045E+01 & -7.6045E+01 & -7.6045E+01  \\ \hline
    Aug-cc-pVTZ & -7.6060E+01 & -7.6074E+01 & -7.6072E+01 & -7.6071E+01 & -7.6070E+01 & -7.6069E+01 & -7.6068E+01 & -7.6067E+01 & -7.6067E+01 & -7.6066E+01 & -7.6066E+01 & -7.6065E+01 & -7.6065E+01 & -7.6064E+01 & -7.6064E+01 & -7.6064E+01  \\ \hline
    Aug-cc-pVQZ & -7.6066E+01 & -7.6079E+01 & -7.6077E+01 & -7.6076E+01 & -7.6075E+01 & -7.6074E+01 & -7.6073E+01 & -7.6073E+01 & -7.6072E+01 & -7.6071E+01 & -7.6071E+01 & -7.6070E+01 & -7.6070E+01 & -7.6070E+01 & -7.6069E+01 & -7.6069E+01  \\ \hline
    Aug-cc-pV5Z & -7.6067E+01 & -7.6080E+01 & -7.6079E+01 & -7.6077E+01 & -7.6076E+01 & -7.6075E+01 & -7.6075E+01 & -7.6074E+01 & -7.6073E+01 & -7.6073E+01 & -7.6072E+01 & -7.6072E+01 & -7.6071E+01 & -7.6071E+01 & -7.6071E+01 & -7.6071E+01  \\ \hline
    daug-cc-pVDZ & -7.6042E+01 & -7.6055E+01 & -7.6053E+01 & -7.6052E+01 & -7.6051E+01 & -7.6050E+01 & -7.6049E+01 & -7.6049E+01 & -7.6048E+01 & -7.6047E+01 & -7.6047E+01 & -7.6046E+01 & -7.6046E+01 & -7.6046E+01 & -7.6045E+01 & -7.6045E+01  \\ \hline
    daug-cc-pVTZ & -7.6061E+01 & -7.6074E+01 & -7.6072E+01 & -7.6071E+01 & -7.6070E+01 & -7.6069E+01 & -7.6068E+01 & -7.6067E+01 & -7.6067E+01 & -7.6066E+01 & -7.6066E+01 & -7.6065E+01 & -7.6065E+01 & -7.6064E+01 & -7.6064E+01 & -7.6064E+01  \\ \hline
    daug-cc-pVQZ & -7.6066E+01 & -7.6079E+01 & -7.6077E+01 & -7.6076E+01 & -7.6075E+01 & -7.6074E+01 & -7.6073E+01 & -7.6073E+01 & -7.6072E+01 & -7.6071E+01 & -7.6071E+01 & -7.6071E+01 & -7.6070E+01 & -7.6070E+01 & -7.6069E+01 & -7.6069E+01 \\ \hline
    daug-cc-pV5Z & -7.6067E+01 & -7.6080E+01 & -7.6079E+01 & -7.6077E+01 & -7.6076E+01 & -7.6075E+01 & -7.6075E+01 & -7.6074E+01 & -7.6073E+01 & -7.6073E+01 & -7.6072E+01 & -7.6072E+01 & -7.6071E+01 & -7.6071E+01 & -7.6071E+01 & -7.6071E+01 \\ \hline
    mrchem & -7.6067E+01 & -7.6085E+01 & -7.6083E+01 & -7.6081E+01 & -7.6079E+01 & -7.6078E+01 & -7.6076E+01 & -7.6075E+01 & -7.6075E+01 & -7.6074E+01 & -7.6073E+01 & -7.6073E+01 & -7.6072E+01 & -7.6072E+01 & -7.6071E+01 & -7.4670E+01  \\ \hline
    variational & -7.6067E+01 & -7.6247E+01 & -7.6160E+01 & -7.6090E+01 & -7.6034E+01 & -7.5989E+01 & -7.5955E+01 & -7.5928E+01 & -7.5908E+01 & -7.5894E+01 & -7.5884E+01 & -7.5878E+01 & -7.5875E+01 & -7.5874E+01 & -7.5875E+01 & -7.4566E+01  \\ \hline
  \end{tabular}}}{\caption{Total Energy of \ce{H_2O}. Radius in top row in Bohr and energies in Hartree}
  \label{tab:rawwatdata}}

\ttabbox{

  \resizebox{\textwidth}{!}{
  \begin{tabular}{|l|r|r|r|r|r|r|r|r|r|r|r|r|r|r|r|r|}
    \hline
    basis & \multicolumn{1}{l|}{vac} & 3.6 & 3.7 & 3.8 & 3.9 & 4 & 4.1 & 4.2 & 4.3 & 4.4 & 4.5 & 4.6 & 4.7 & 4.8 & 4.9 & 5 \\ \hline
    Cc-pVDZ & -1.2893E+02 & -1.2907E+02 & -1.2906E+02 & -1.2906E+02 & -1.2906E+02 & -1.2905E+02 & -1.2905E+02 & -1.2905E+02 & -1.2904E+02 & -1.2904E+02 & -1.2904E+02 & -1.2904E+02 & -1.2903E+02 & -1.2903E+02 & -1.2903E+02 & -1.2903E+02 \\ \hline
    Cc-pVTZ & -1.2897E+02 & -1.2911E+02 & -1.2910E+02 & -1.2910E+02 & -1.2910E+02 & -1.2909E+02 & -1.2909E+02 & -1.2909E+02 & -1.2908E+02 & -1.2908E+02 & -1.2908E+02 & -1.2907E+02 & -1.2907E+02 & -1.2907E+02 & -1.2907E+02 & -1.2907E+02 \\ \hline
    Cc-pVQZ & -1.2898E+02 & -1.2912E+02 & -1.2911E+02 & -1.2911E+02 & -1.2911E+02 & -1.2910E+02 & -1.2910E+02 & -1.2910E+02 & -1.2909E+02 & -1.2909E+02 & -1.2909E+02 & -1.2908E+02 & -1.2908E+02 & -1.2908E+02 & -1.2908E+02 & -1.2908E+02 \\ \hline
    Cc-pV5Z & -1.2898E+02 & -1.2912E+02 & -1.2912E+02 & -1.2911E+02 & -1.2911E+02 & -1.2910E+02 & -1.2910E+02 & -1.2910E+02 & -1.2909E+02 & -1.2909E+02 & -1.2909E+02 & -1.2909E+02 & -1.2908E+02 & -1.2908E+02 & -1.2908E+02 & -1.2908E+02 \\ \hline
    Aug-cc-pVDZ & -1.2894E+02 & -1.2908E+02 & -1.2907E+02 & -1.2907E+02 & -1.2906E+02 & -1.2906E+02 & -1.2906E+02 & -1.2905E+02 & -1.2905E+02 & -1.2905E+02 & -1.2905E+02 & -1.2904E+02 & -1.2904E+02 & -1.2904E+02 & -1.2904E+02 & -1.2903E+02 \\ \hline
    Aug-cc-pVTZ & -1.2897E+02 & -1.2911E+02 & -1.2910E+02 & -1.2910E+02 & -1.2910E+02 & -1.2909E+02 & -1.2909E+02 & -1.2909E+02 & -1.2908E+02 & -1.2908E+02 & -1.2908E+02 & -1.2908E+02 & -1.2907E+02 & -1.2907E+02 & -1.2907E+02 & -1.2907E+02 \\ \hline
    Aug-cc-pVQZ & -1.2898E+02 & -1.2912E+02 & -1.2911E+02 & -1.2911E+02 & -1.2911E+02 & -1.2910E+02 & -1.2910E+02 & -1.2910E+02 & -1.2909E+02 & -1.2909E+02 & -1.2909E+02 & -1.2909E+02 & -1.2908E+02 & -1.2908E+02 & -1.2908E+02 & -1.2908E+02 \\ \hline
    Aug-cc-pV5Z & -1.2898E+02 & -1.2912E+02 & -1.2912E+02 & -1.2911E+02 & -1.2911E+02 & -1.2910E+02 & -1.2910E+02 & -1.2910E+02 & -1.2909E+02 & -1.2909E+02 & -1.2909E+02 & -1.2909E+02 & -1.2908E+02 & -1.2908E+02 & -1.2908E+02 & -1.2908E+02 \\ \hline
    daug-cc-pVDZ & -1.2894E+02 & -1.2908E+02 & -1.2907E+02 & -1.2907E+02 & -1.2906E+02 & -1.2906E+02 & -1.2906E+02 & -1.2905E+02 & -1.2905E+02 & -1.2905E+02 & -1.2905E+02 & -1.2904E+02 & -1.2904E+02 & -1.2904E+02 & -1.2904E+02 & -1.2904E+02 \\ \hline
    daug-cc-pVTZ & -1.2897E+02 & -1.2911E+02 & -1.2910E+02 & -1.2910E+02 & -1.2910E+02 & -1.2909E+02 & -1.2909E+02 & -1.2909E+02 & -1.2908E+02 & -1.2908E+02 & -1.2908E+02 & -1.2908E+02 & -1.2907E+02 & -1.2907E+02 & -1.2907E+02 & -1.2907E+02 \\ \hline
    daug-cc-pVQZ & -1.2898E+02 & -1.2912E+02 & -1.2911E+02 & -1.2911E+02 & -1.2911E+02 & -1.2910E+02 & -1.2910E+02 & -1.2910E+02 & -1.2909E+02 & -1.2909E+02 & -1.2909E+02 & -1.2909E+02 & -1.2908E+02 & -1.2908E+02 & -1.2908E+02 & -1.2908E+02 \\ \hline
    daug-cc-pV5Z & -1.2898E+02 & -1.2912E+02 & -1.2912E+02 & -1.2911E+02 & -1.2911E+02 & -1.2910E+02 & -1.2910E+02 & -1.2910E+02 & -1.2909E+02 & -1.2909E+02 & -1.2909E+02 & -1.2909E+02 & -1.2908E+02 & -1.2908E+02 & -1.2908E+02 & -1.2908E+02 \\ \hline
    mrchem & -1.2898E+02 & -1.2931E+02 & -1.2924E+02 & -1.2920E+02 & -1.2917E+02 & -1.2915E+02 & -1.2566E+02 & \multicolumn{1}{l|}{N/A} & -1.2565E+02 & -1.2565E+02 & -1.2911E+02 & \multicolumn{1}{l|}{N/A} & -1.2564E+02 & -1.2909E+02 & -1.2909E+02 & -1.2909E+02 \\ \hline
    variational & -1.2898E+02 & -1.2794E+02 & -1.2781E+02 & -1.2789E+02 & -1.2808E+02 & -1.2830E+02 & -1.2566E+02 & \multicolumn{1}{l|}{N/A} & -1.2565E+02 & -1.2565E+02 & -1.2900E+02 & -1.2905E+02 & -1.2564E+02 & -1.2909E+02 & -1.2910E+02 & -1.2910E+02 \\ \hline
  \end{tabular}}}{\caption{Total Energy of \ce{NO^+}.  Radius in top row in Bohr and energies in Hartree}
  \label{tab:rawnopdata}}


\ttabbox{

\resizebox{\textwidth}{!}{
\begin{tabular}{|l|r|r|r|r|r|r|r|r|r|r|r|r|r|r|r|r|}
\hline
basis & \multicolumn{1}{l|}{vac} & 3.6 & 3.7 & 3.8 & 3.9 & 4 & 4.1 & 4.2 & 4.3 & 4.4 & 4.5 & 4.6 & 4.7 & 4.8 & 4.9 & 5 \\ \hline
Cc-pVDZ & -7.6027E+01 & -9.2425E+01 & -9.2425E+01 & -9.2423E+01 & -9.2422E+01 & -9.2420E+01 & -9.2418E+01 & -9.2416E+01 & -9.2414E+01 & -9.2412E+01 & -9.2410E+01 & -9.2408E+01 & -9.2405E+01 & -9.2403E+01 & -9.2401E+01 & -9.2399E+01  \\ \hline
Cc-pVTZ & -7.6057E+01 & -9.2455E+01 & -9.2455E+01 & -9.2454E+01 & -9.2453E+01 & -9.2452E+01 & -9.2451E+01 & -9.2449E+01 & -9.2447E+01 & -9.2445E+01 & -9.2443E+01 & -9.2441E+01 & -9.2439E+01 & -9.2437E+01 & -9.2435E+01 & -9.2433E+01  \\ \hline
Cc-pVQZ & -7.6065E+01 & -9.2464E+01 & -9.2463E+01 & -9.2463E+01 & -9.2462E+01 & -9.2461E+01 & -9.2460E+01 & -9.2458E+01 & -9.2456E+01 & -9.2455E+01 & -9.2453E+01 & -9.2451E+01 & -9.2449E+01 & -9.2447E+01 & -9.2445E+01 & -9.2443E+01  \\ \hline
Cc-pV5Z & -7.6067E+01 & -9.2465E+01 & -9.2465E+01 & -9.2465E+01 & -9.2464E+01 & -9.2463E+01 & -9.2462E+01 & -9.2460E+01 & -9.2459E+01 & -9.2457E+01 & -9.2455E+01 & -9.2454E+01 & -9.2452E+01 & -9.2450E+01 & -9.2448E+01 & -9.2446E+01  \\ \hline
Aug-cc-pVDZ & -7.6041E+01 & -9.2441E+01 & -9.2441E+01 & -9.2440E+01 & -9.2439E+01 & -9.2438E+01 & -9.2437E+01 & -9.2436E+01 & -9.2434E+01 & -9.2433E+01 & -9.2431E+01 & -9.2429E+01 & -9.2427E+01 & -9.2426E+01 & -9.2424E+01 & -9.2422E+01  \\ \hline
Aug-cc-pVTZ & -7.6060E+01 & -9.2459E+01 & -9.2459E+01 & -9.2459E+01 & -9.2458E+01 & -9.2457E+01 & -9.2456E+01 & -9.2454E+01 & -9.2453E+01 & -9.2451E+01 & -9.2449E+01 & -9.2448E+01 & -9.2446E+01 & -9.2444E+01 & -9.2442E+01 & -9.2441E+01  \\ \hline
Aug-cc-pVQZ & -7.6066E+01 & -9.2465E+01 & -9.2465E+01 & -9.2464E+01 & -9.2463E+01 & -9.2462E+01 & -9.2461E+01 & -9.2460E+01 & -9.2458E+01 & -9.2456E+01 & -9.2455E+01 & -9.2453E+01 & -9.2451E+01 & -9.2449E+01 & -9.2448E+01 & -9.2446E+01  \\ \hline
Aug-cc-pV5Z & -7.6067E+01 & -9.2466E+01 & -9.2466E+01 & -9.2465E+01 & -9.2464E+01 & -9.2463E+01 & -9.2462E+01 & -9.2461E+01 & -9.2459E+01 & -9.2457E+01 & -9.2456E+01 & -9.2454E+01 & -9.2452E+01 & -9.2450E+01 & -9.2449E+01 & -9.2447E+01  \\ \hline
daug-cc-pVDZ & -7.6042E+01 & -9.2441E+01 & -9.2441E+01 & -9.2440E+01 & -9.2440E+01 & -9.2439E+01 & -9.2437E+01 & -9.2436E+01 & -9.2435E+01 & -9.2433E+01 & -9.2431E+01 & -9.2430E+01 & -9.2428E+01 & -9.2426E+01 & -9.2424E+01 & -9.2423E+01  \\ \hline
daug-cc-pVTZ & -7.6061E+01 & -9.2459E+01 & -9.2459E+01 & -9.2459E+01 & -9.2458E+01 & -9.2457E+01 & -9.2456E+01 & -9.2454E+01 & -9.2453E+01 & -9.2451E+01 & -9.2449E+01 & -9.2448E+01 & -9.2446E+01 & -9.2444E+01 & -9.2443E+01 & -9.2441E+01  \\ \hline
daug-cc-pVQZ & -7.6066E+01 & -9.2465E+01 & -9.2465E+01 & -9.2464E+01 & -9.2463E+01 & -9.2462E+01 & -9.2461E+01 & -9.2460E+01 & -9.2458E+01 & -9.2456E+01 & -9.2455E+01 & -9.2453E+01 & -9.2451E+01 & -9.2449E+01 & -9.2448E+01 & -9.2446E+01  \\ \hline
daug-cc-pV5Z & -7.6067E+01 & -9.2466E+01 & -9.2466E+01 & -9.2465E+01 & -9.2464E+01 & -9.2463E+01 & -9.2462E+01 & -9.2461E+01 & -9.2459E+01 & -9.2457E+01 & -9.2456E+01 & -9.2454E+01 & -9.2452E+01 & -9.2450E+01 & -9.2449E+01 & -9.2447E+01  \\ \hline
mrchem & -9.2349E+01 & -9.2472E+01 & -9.2471E+01 & -9.2470E+01 & -9.2469E+01 & -8.9965E+01 & -9.2466E+01 & \multicolumn{1}{l|}{N/A} & -9.2463E+01 & \multicolumn{1}{l|}{N/A} & -8.9954E+01 & -9.2458E+01 & \multicolumn{1}{l|}{N/A} & -9.2454E+01 & -8.9945E+01 & -9.2450E+01  \\ \hline
variational & -9.2349E+01 & -9.7694E+01 & -9.7347E+01 & -9.6991E+01 & -9.6630E+01 & -9.6271E+01 & -9.5920E+01 & \multicolumn{1}{l|}{N/A} & -9.5263E+01 & \multicolumn{1}{l|}{N/A} & -9.4678E+01 & -9.4432E+01 & -9.4199E+01 & -9.3990E+01 & -9.3801E+01 & -9.3632E+01  \\ \hline
\end{tabular}}} {\caption{Total Energy of \ce{CN^-}.  Radius in top row in Bohr and energies in Hartree}
\label{tab:rawcyandata}}

\ttabbox{

\resizebox{\textwidth}{!}{

\begin{tabular}{|l|r|r|r|r|r|r|r|r|r|r|r|r|r|r|r|r|}
\hline
basis & \multicolumn{1}{l|}{vac} & 3.6 & 3.7 & 3.8 & 3.9 & 4 & 4.1 & 4.2 & 4.3 & 4.4 & 4.5 & 4.6 & 4.7 & 4.8 & 4.9 & 5 \\ \hline
Cc-pVDZ & -7.2361E+00 & -7.3732E+00 & -7.3695E+00 & -7.3660E+00 & -7.3627E+00 & -7.3595E+00 & -7.3565E+00 & -7.3536E+00 & -7.3509E+00 & -7.3483E+00 & -7.3458E+00 & -7.3434E+00 & -7.3411E+00 & -7.3390E+00 & -7.3369E+00 & -7.3348E+00 \\ \hline
Cc-pVTZ & -7.2364E+00 & -7.3735E+00 & -7.3698E+00 & -7.3663E+00 & -7.3629E+00 & -7.3598E+00 & -7.3568E+00 & -7.3539E+00 & -7.3512E+00 & -7.3486E+00 & -7.3461E+00 & -7.3437E+00 & -7.3414E+00 & -7.3392E+00 & -7.3371E+00 & -7.3351E+00 \\ \hline
Cc-pVQZ & -7.2364E+00 & -7.3735E+00 & -7.3698E+00 & -7.3663E+00 & -7.3630E+00 & -7.3598E+00 & -7.3568E+00 & -7.3539E+00 & -7.3512E+00 & -7.3486E+00 & -7.3461E+00 & -7.3437E+00 & -7.3414E+00 & -7.3392E+00 & -7.3371E+00 & -7.3351E+00 \\ \hline
Cc-pV5Z & -7.2364E+00 & -7.3735E+00 & -7.3698E+00 & -7.3663E+00 & -7.3630E+00 & -7.3598E+00 & -7.3568E+00 & -7.3539E+00 & -7.3512E+00 & -7.3486E+00 & -7.3461E+00 & -7.3437E+00 & -7.3414E+00 & -7.3392E+00 & -7.3371E+00 & -7.3351E+00 \\ \hline
Aug-cc-pVDZ & -7.2361E+00 & -7.3732E+00 & -7.3695E+00 & -7.3660E+00 & -7.3627E+00 & -7.3595E+00 & -7.3565E+00 & -7.3536E+00 & -7.3509E+00 & -7.3483E+00 & -7.3458E+00 & -7.3434E+00 & -7.3411E+00 & -7.3390E+00 & -7.3369E+00 & -7.3348E+00 \\ \hline
Aug-cc-pVTZ & -7.2364E+00 & -7.3735E+00 & -7.3698E+00 & -7.3663E+00 & -7.3629E+00 & -7.3598E+00 & -7.3568E+00 & -7.3539E+00 & -7.3512E+00 & -7.3486E+00 & -7.3461E+00 & -7.3437E+00 & -7.3414E+00 & -7.3392E+00 & -7.3371E+00 & -7.3351E+00 \\ \hline
Aug-cc-pVQZ & -7.2364E+00 & -7.3735E+00 & -7.3698E+00 & -7.3663E+00 & -7.3630E+00 & -7.3598E+00 & -7.3568E+00 & -7.3539E+00 & -7.3512E+00 & -7.3486E+00 & -7.3461E+00 & -7.3437E+00 & -7.3414E+00 & -7.3392E+00 & -7.3371E+00 & -7.3351E+00 \\ \hline
Aug-cc-pV5Z & -7.2364E+00 & -7.3735E+00 & -7.3698E+00 & -7.3663E+00 & -7.3630E+00 & -7.3598E+00 & -7.3568E+00 & -7.3539E+00 & -7.3512E+00 & -7.3486E+00 & -7.3461E+00 & -7.3437E+00 & -7.3414E+00 & -7.3392E+00 & -7.3371E+00 & -7.3351E+00 \\ \hline
daug-cc-pVDZ & -7.2361E+00 & -7.3732E+00 & -7.3695E+00 & -7.3660E+00 & -7.3627E+00 & -7.3595E+00 & -7.3565E+00 & -7.3536E+00 & -7.3509E+00 & -7.3483E+00 & -7.3458E+00 & -7.3434E+00 & -7.3411E+00 & -7.3390E+00 & -7.3369E+00 & -7.3348E+00 \\ \hline
daug-cc-pVTZ & -7.2364E+00 & -7.3735E+00 & -7.3698E+00 & -7.3663E+00 & -7.3629E+00 & -7.3598E+00 & -7.3568E+00 & -7.3539E+00 & -7.3512E+00 & -7.3486E+00 & -7.3461E+00 & -7.3437E+00 & -7.3414E+00 & -7.3392E+00 & -7.3371E+00 & -7.3351E+00 \\ \hline
daug-cc-pVQZ & -7.2364E+00 & -7.3735E+00 & -7.3698E+00 & -7.3663E+00 & -7.3630E+00 & -7.3598E+00 & -7.3568E+00 & -7.3539E+00 & -7.3512E+00 & -7.3486E+00 & -7.3461E+00 & -7.3437E+00 & -7.3414E+00 & -7.3392E+00 & -7.3371E+00 & -7.3351E+00 \\ \hline
daug-cc-pV5Z & -7.2364E+00 & -7.3735E+00 & -7.3698E+00 & -7.3663E+00 & -7.3630E+00 & -7.3598E+00 & -7.3568E+00 & -7.3539E+00 & -7.3512E+00 & -7.3486E+00 & -7.3461E+00 & -7.3437E+00 & -7.3414E+00 & -7.3392E+00 & -7.3371E+00 & -7.3351E+00 \\ \hline
mrchem & -7.2364E+00 & -7.3792E+00 & -7.3752E+00 & -7.3714E+00 & -7.3678E+00 & -7.3644E+00 & -7.3612E+00 & -7.3581E+00 & -7.3551E+00 & -7.3524E+00 & -7.3497E+00 & -7.3472E+00 & -7.3447E+00 & -7.3424E+00 & -7.3402E+00 & -7.3380E+00 \\ \hline
variational & -7.2364E+00 & -7.3744E+00 & -7.3707E+00 & -7.3671E+00 & -7.3636E+00 & -7.3603E+00 & -7.3572E+00 & -7.3542E+00 & -7.3514E+00 & -7.3487E+00 & -7.3461E+00 & -7.3438E+00 & -7.3415E+00 & -7.3393E+00 & -7.3372E+00 & -7.3352E+00 \\ \hline
\end{tabular}}}{\caption{Total energy of \ce{Li^+}. Radius on top row in Bohr and energies in Hartree}
\label{tab:rawLipdata}}

\end{sidewaystable}

\begin{sidewaystable}
  \ttabbox{
  \resizebox{\textwidth}{!}{
    \begin{tabular}{|l|r|r|r|r|r|r|r|r|r|r|r|r|r|r|r|}
    \hline
    basis & 3.6 & 3.7 & 3.8 & 3.9 & 4 & 4.1 & 4.2 & 4.3 & 4.4 & 4.5 & 4.6 & 4.7 & 4.8 & 4.9 & 5 \\ \hline
    Cc-pVDZ & -1.2450E-02 & -1.0998E-02 & -9.7804E-03 & -8.7499E-03 & -7.8700E-03 & -7.1128E-03 & -6.4573E-03 & -5.8844E-03 & -5.3816E-03 & -4.9378E-03 & -4.5442E-03 & -4.1936E-03 & -3.8799E-03 & -3.5983E-03 & -3.3447E-03 \\ \hline
    Cc-pVTZ & -1.3097E-02 & -1.1545E-02 & -1.0243E-02 & -9.1412E-03 & -8.2007E-03 & -7.3922E-03 & -6.6939E-03 & -6.0843E-03 & -5.5504E-03 & -5.0804E-03 & -4.6646E-03 & -4.2951E-03 & -3.9654E-03 & -3.6700E-03 & -3.4049E-03 \\ \hline
    Cc-pVQZ & -1.3218E-02 & -1.1651E-02 & -1.0334E-02 & -9.2192E-03 & -8.2670E-03 & -7.4482E-03 & -6.7410E-03 & -6.1235E-03 & -5.5827E-03 & -5.1067E-03 & -4.6857E-03 & -4.3118E-03 & -3.9782E-03 & -3.6796E-03 & -3.4117E-03 \\ \hline
    Cc-pV5Z & -1.3284E-02 & -1.1713E-02 & -1.0393E-02 & -9.2737E-03 & -8.3174E-03 & -7.4945E-03 & -6.7834E-03 & -6.1620E-03 & -5.6175E-03 & -5.1380E-03 & -4.7138E-03 & -4.3369E-03 & -4.0007E-03 & -3.6996E-03 & -3.4295E-03 \\ \hline
    Aug-cc-pVDZ & -1.3190E-02 & -1.1634E-02 & -1.0328E-02 & -9.2204E-03 & -8.2741E-03 & -7.4596E-03 & -6.7553E-03 & -6.1396E-03 & -5.5997E-03 & -5.1238E-03 & -4.7025E-03 & -4.3279E-03 & -3.9935E-03 & -3.6939E-03 & -3.4249E-03 \\ \hline
    Aug-cc-pVTZ & -1.3238E-02 & -1.1670E-02 & -1.0353E-02 & -9.2358E-03 & -8.2817E-03 & -7.4605E-03 & -6.7510E-03 & -6.1309E-03 & -5.5876E-03 & -5.1091E-03 & -4.6858E-03 & -4.3097E-03 & -3.9742E-03 & -3.6739E-03 & -3.4045E-03 \\ \hline
    Aug-cc-pVQZ & -1.3221E-02 & -1.1655E-02 & -1.0338E-02 & -9.2227E-03 & -8.2695E-03 & -7.4493E-03 & -6.7407E-03 & -6.1214E-03 & -5.5789E-03 & -5.1011E-03 & -4.6784E-03 & -4.3029E-03 & -3.9680E-03 & -3.6681E-03 & -3.3992E-03 \\ \hline
    Aug-cc-pV5Z & -1.3223E-02 & -1.1655E-02 & -1.0337E-02 & -9.2207E-03 & -8.2672E-03 & -7.4469E-03 & -6.7383E-03 & -6.1192E-03 & -5.5769E-03 & -5.0994E-03 & -4.6770E-03 & -4.3018E-03 & -3.9672E-03 & -3.6676E-03 & -3.3988E-03 \\ \hline
    daug-cc-pVDZ & -1.3228E-02 & -1.1665E-02 & -1.0351E-02 & -9.2383E-03 & -8.2871E-03 & -7.4682E-03 & -6.7605E-03 & -6.1417E-03 & -5.5992E-03 & -5.1213E-03 & -4.6983E-03 & -4.3223E-03 & -3.9868E-03 & -3.6863E-03 & -3.4166E-03 \\ \hline
    daug-cc-pVTZ & -1.3243E-02 & -1.1675E-02 & -1.0357E-02 & -9.2398E-03 & -8.2849E-03 & -7.4631E-03 & -6.7529E-03 & -6.1322E-03 & -5.5882E-03 & -5.1092E-03 & -4.6855E-03 & -4.3091E-03 & -3.9734E-03 & -3.6729E-03 & -3.4034E-03 \\ \hline
    daug-cc-pVQZ & -1.3223E-02 & -1.1656E-02 & -1.0340E-02 & -9.2243E-03 & -8.2711E-03 & -7.4508E-03 & -6.7421E-03 & -6.1227E-03 & -5.5800E-03 & -5.1021E-03 & -4.6793E-03 & -4.3037E-03 & -3.9686E-03 & -3.6687E-03 & -3.3997E-03 \\ \hline
    daug-cc-pV5Z & -1.3224E-02 & -1.1655E-02 & -1.0337E-02 & -9.2210E-03 & -8.2674E-03 & -7.4470E-03 & -6.7385E-03 & -6.1193E-03 & -5.5770E-03 & -5.0995E-03 & -4.6771E-03 & -4.3019E-03 & -3.9672E-03 & -3.6676E-03 & -3.3988E-03 \\ \hline
    mrchem & -1.8036E-02 & -1.5494E-02 & -1.3437E-02 & -1.1751E-02 & -1.0354E-02 & -9.1865E-03 & -8.1992E-03 & -7.3577E-03 & -6.6339E-03 & -6.0092E-03 & -5.4647E-03 & -4.9879E-03 & -4.5682E-03 & -4.1968E-03 & 1.3975E+00 \\ \hline
    variational & -1.7939E-01 & -9.2889E-02 & -2.2807E-02 & 3.3437E-02 & 7.7993E-02 & 1.1246E-01 & 1.3945E-01 & 1.5927E-01 & 1.7367E-01 & 1.8347E-01 & 1.8957E-01 & 1.9266E-01 & 1.9338E-01 & 1.9217E-01 & 1.5012E+00 \\ \hline
    \end{tabular}}}{  \caption{Reaction field energy of \ce{H_2O}. Radius on top row in Bohr and energies in Hartree}\label{tab:Erdatawat}}

  \ttabbox{
  \resizebox{\textwidth}{!}{
  \begin{tabular}{|l|r|r|r|r|r|r|r|r|r|r|r|r|r|r|r|r|r|}
  \hline
  basis & 3.6 & 3.7 & 3.8 & 3.9 & 4 & 4.1 & 4.2 & 4.3 & 4.4 & 4.5 & 4.6 & 4.7 & 4.8 & 4.9 & 5 & \multicolumn{1}{l|}{} & \multicolumn{1}{l|}{} \\ \hline
  Cc-pVDZ & -1.4058E-01 & -1.3613E-01 & -1.3206E-01 & -1.2830E-01 & -1.2481E-01 & -1.2155E-01 & -1.1849E-01 & -1.1561E-01 & -1.1289E-01 & -1.1031E-01 & -1.0785E-01 & -1.0551E-01 & -1.0327E-01 & -1.0114E-01 & -9.9088E-02 & -1.2929E-01 & -1.2239E-01 \\ \hline
  Cc-pVTZ & -1.4064E-01 & -1.3621E-01 & -1.3214E-01 & -1.2839E-01 & -1.2489E-01 & -1.2162E-01 & -1.1854E-01 & -1.1565E-01 & -1.1291E-01 & -1.1032E-01 & -1.0785E-01 & -1.0550E-01 & -1.0326E-01 & -1.0112E-01 & -9.9067E-02 & -1.2916E-01 & -1.2227E-01 \\ \hline
  Cc-pVQZ & -1.4042E-01 & -1.3604E-01 & -1.3202E-01 & -1.2829E-01 & -1.2482E-01 & -1.2157E-01 & -1.1851E-01 & -1.1563E-01 & -1.1290E-01 & -1.1031E-01 & -1.0785E-01 & -1.0550E-01 & -1.0326E-01 & -1.0112E-01 & -9.9068E-02 & -1.2909E-01 & -1.2223E-01 \\ \hline
  Cc-pV5Z & -1.4032E-01 & -1.3595E-01 & -1.3193E-01 & -1.2821E-01 & -1.2475E-01 & -1.2151E-01 & -1.1846E-01 & -1.1559E-01 & -1.1287E-01 & -1.1028E-01 & -1.0783E-01 & -1.0549E-01 & -1.0325E-01 & -1.0111E-01 & -9.9059E-02 & -1.2900E-01 & -1.2218E-01 \\ \hline
  Aug-cc-pVDZ & -1.4051E-01 & -1.3606E-01 & -1.3200E-01 & -1.2824E-01 & -1.2476E-01 & -1.2150E-01 & -1.1844E-01 & -1.1556E-01 & -1.1284E-01 & -1.1026E-01 & -1.0780E-01 & -1.0546E-01 & -1.0323E-01 & -1.0109E-01 & -9.9045E-02 & -1.2887E-01 & -1.2206E-01 \\ \hline
  Aug-cc-pVTZ & -1.4048E-01 & -1.3608E-01 & -1.3203E-01 & -1.2829E-01 & -1.2481E-01 & -1.2156E-01 & -1.1849E-01 & -1.1561E-01 & -1.1288E-01 & -1.1029E-01 & -1.0783E-01 & -1.0549E-01 & -1.0325E-01 & -1.0111E-01 & -9.9056E-02 & -1.2902E-01 & -1.2216E-01 \\ \hline
  Aug-cc-pVQZ & -1.4036E-01 & -1.3598E-01 & -1.3196E-01 & -1.2824E-01 & -1.2478E-01 & -1.2153E-01 & -1.1848E-01 & -1.1560E-01 & -1.1288E-01 & -1.1029E-01 & -1.0783E-01 & -1.0549E-01 & -1.0325E-01 & -1.0111E-01 & -9.9059E-02 & -1.2903E-01 & -1.2218E-01 \\ \hline
  Aug-cc-pV5Z & -1.4032E-01 & -1.3594E-01 & -1.3193E-01 & -1.2821E-01 & -1.2475E-01 & -1.2151E-01 & -1.1846E-01 & -1.1558E-01 & -1.1286E-01 & -1.1028E-01 & -1.0783E-01 & -1.0548E-01 & -1.0325E-01 & -1.0111E-01 & -9.9058E-02 & -1.2900E-01 & -1.2217E-01 \\ \hline
  daug-cc-pVDZ & -1.4043E-01 & -1.3601E-01 & -1.3196E-01 & -1.2823E-01 & -1.2475E-01 & -1.2150E-01 & -1.1845E-01 & -1.1558E-01 & -1.1286E-01 & -1.1028E-01 & -1.0783E-01 & -1.0549E-01 & -1.0325E-01 & -1.0112E-01 & -9.9068E-02 & -1.2889E-01 & -1.2208E-01 \\ \hline
  daug-cc-pVTZ & -1.4040E-01 & -1.3601E-01 & -1.3198E-01 & -1.2825E-01 & -1.2478E-01 & -1.2153E-01 & -1.1847E-01 & -1.1560E-01 & -1.1287E-01 & -1.1029E-01 & -1.0783E-01 & -1.0549E-01 & -1.0325E-01 & -1.0111E-01 & -9.9059E-02 & -1.2900E-01 & -1.2216E-01 \\ \hline
  daug-cc-pVQZ & -1.4034E-01 & -1.3596E-01 & -1.3194E-01 & -1.2822E-01 & -1.2476E-01 & -1.2152E-01 & -1.1846E-01 & -1.1559E-01 & -1.1287E-01 & -1.1028E-01 & -1.0783E-01 & -1.0548E-01 & -1.0325E-01 & -1.0111E-01 & -9.9058E-02 & -1.2902E-01 & -1.2218E-01 \\ \hline
  daug-cc-pV5Z & -1.4032E-01 & -1.3594E-01 & -1.3193E-01 & -1.2821E-01 & -1.2475E-01 & -1.2151E-01 & -1.1846E-01 & -1.1558E-01 & -1.1286E-01 & -1.1028E-01 & -1.0782E-01 & -1.0548E-01 & -1.0325E-01 & -1.0111E-01 & -9.9058E-02 & -1.2900E-01 & -1.2217E-01 \\ \hline
  mrchem & -3.2883E-01 & -2.5954E-01 & -2.1649E-01 & -1.8860E-01 & -1.6969E-01 & 3.3235E+00 & N/A & 3.3308E+00 & 3.3341E+00 & -1.2727E-01 & N/A & 3.3429E+00 & -1.1538E-01 & -1.1218E-01 & -1.0924E-01 & -1.3411E-01 & -1.2670E-01 \\ \hline
  variational & 1.0359E+00 & 1.1729E+00 & 1.0849E+00 & 8.9494E-01 & 6.7670E-01 & 3.3235E+00 & N/A & 3.3308E+00 & 3.3341E+00 & -2.0407E-02 & -6.8498E-02 & 3.3429E+00 & -1.1235E-01 & -1.1730E-01 & -1.1578E-01 & -1.8326E-01 & -1.5337E-01 \\ \hline
  \end{tabular}}}{\caption{Reaction field energy of \ce{NO^+}. Radius on top row in Bohr and energies in Hartree}
  \label{tab:Erdatanop}}

  \ttabbox{
  \resizebox{\textwidth}{!}{
\begin{tabular}{|l|r|r|r|r|r|r|r|r|r|r|r|r|r|r|r|}
\hline
basis & 3.6 & 3.7 & 3.8 & 3.9 & 4 & 4.1 & 4.2 & 4.3 & 4.4 & 4.5 & 4.6 & 4.7 & 4.8 & 4.9 & 5 \\ \hline
Cc-pVDZ & -1.6399E+01 & -1.6398E+01 & -1.6397E+01 & -1.6395E+01 & -1.6394E+01 & -1.6392E+01 & -1.6390E+01 & -1.6388E+01 & -1.6385E+01 & -1.6383E+01 & -1.6381E+01 & -1.6379E+01 & -1.6377E+01 & -1.6374E+01 & -1.6372E+01 \\ \hline
Cc-pVTZ & -1.6398E+01 & -1.6398E+01 & -1.6398E+01 & -1.6396E+01 & -1.6395E+01 & -1.6394E+01 & -1.6392E+01 & -1.6390E+01 & -1.6388E+01 & -1.6386E+01 & -1.6384E+01 & -1.6382E+01 & -1.6380E+01 & -1.6378E+01 & -1.6376E+01 \\ \hline
Cc-pVQZ & -1.6399E+01 & -1.6399E+01 & -1.6398E+01 & -1.6397E+01 & -1.6396E+01 & -1.6395E+01 & -1.6393E+01 & -1.6392E+01 & -1.6390E+01 & -1.6388E+01 & -1.6386E+01 & -1.6384E+01 & -1.6383E+01 & -1.6381E+01 & -1.6379E+01 \\ \hline
Cc-pV5Z & -1.6399E+01 & -1.6399E+01 & -1.6398E+01 & -1.6397E+01 & -1.6396E+01 & -1.6395E+01 & -1.6393E+01 & -1.6392E+01 & -1.6390E+01 & -1.6388E+01 & -1.6387E+01 & -1.6385E+01 & -1.6383E+01 & -1.6381E+01 & -1.6380E+01 \\ \hline
Aug-cc-pVDZ & -1.6399E+01 & -1.6399E+01 & -1.6399E+01 & -1.6398E+01 & -1.6397E+01 & -1.6396E+01 & -1.6395E+01 & -1.6393E+01 & -1.6391E+01 & -1.6390E+01 & -1.6388E+01 & -1.6386E+01 & -1.6384E+01 & -1.6383E+01 & -1.6381E+01 \\ \hline
Aug-cc-pVTZ & -1.6399E+01 & -1.6399E+01 & -1.6398E+01 & -1.6398E+01 & -1.6396E+01 & -1.6395E+01 & -1.6394E+01 & -1.6392E+01 & -1.6391E+01 & -1.6389E+01 & -1.6387E+01 & -1.6385E+01 & -1.6384E+01 & -1.6382E+01 & -1.6380E+01 \\ \hline
Aug-cc-pVQZ & -1.6399E+01 & -1.6399E+01 & -1.6398E+01 & -1.6398E+01 & -1.6396E+01 & -1.6395E+01 & -1.6394E+01 & -1.6392E+01 & -1.6391E+01 & -1.6389E+01 & -1.6387E+01 & -1.6385E+01 & -1.6384E+01 & -1.6382E+01 & -1.6380E+01 \\ \hline
Aug-cc-pV5Z & -1.6399E+01 & -1.6399E+01 & -1.6398E+01 & -1.6397E+01 & -1.6396E+01 & -1.6395E+01 & -1.6394E+01 & -1.6392E+01 & -1.6390E+01 & -1.6389E+01 & -1.6387E+01 & -1.6385E+01 & -1.6383E+01 & -1.6382E+01 & -1.6380E+01 \\ \hline
daug-cc-pVDZ & -1.6399E+01 & -1.6399E+01 & -1.6399E+01 & -1.6398E+01 & -1.6397E+01 & -1.6396E+01 & -1.6394E+01 & -1.6393E+01 & -1.6391E+01 & -1.6390E+01 & -1.6388E+01 & -1.6386E+01 & -1.6384E+01 & -1.6383E+01 & -1.6381E+01 \\ \hline
daug-cc-pVTZ & -1.6399E+01 & -1.6399E+01 & -1.6398E+01 & -1.6398E+01 & -1.6397E+01 & -1.6395E+01 & -1.6394E+01 & -1.6392E+01 & -1.6391E+01 & -1.6389E+01 & -1.6387E+01 & -1.6385E+01 & -1.6384E+01 & -1.6382E+01 & -1.6380E+01 \\ \hline
daug-cc-pVQZ & -1.6399E+01 & -1.6399E+01 & -1.6398E+01 & -1.6398E+01 & -1.6396E+01 & -1.6395E+01 & -1.6394E+01 & -1.6392E+01 & -1.6391E+01 & -1.6389E+01 & -1.6387E+01 & -1.6385E+01 & -1.6384E+01 & -1.6382E+01 & -1.6380E+01 \\ \hline
daug-cc-pV5Z & -1.6399E+01 & -1.6399E+01 & -1.6398E+01 & -1.6397E+01 & -1.6396E+01 & -1.6395E+01 & -1.6394E+01 & -1.6392E+01 & -1.6390E+01 & -1.6389E+01 & -1.6387E+01 & -1.6385E+01 & -1.6383E+01 & -1.6382E+01 & -1.6380E+01 \\ \hline
mrchem & -1.2255E-01 & -1.2182E-01 & -1.2094E-01 & -1.1987E-01 & 2.3842E+00 & -1.1719E-01 & \#VALUE! & -1.1394E-01 & \#VALUE! & 2.3956E+00 & -1.0850E-01 & \#VALUE! & -1.0478E-01 & 2.4041E+00 & -1.0106E-01 \\ \hline
variational & -5.3447E+00 & -4.9982E+00 & -4.6420E+00 & -4.2809E+00 & -3.9219E+00 & -3.5712E+00 & \#VALUE! & -2.9143E+00 & \#VALUE! & -2.3285E+00 & -2.0824E+00 & -1.8503E+00 & -1.6404E+00 & -1.4514E+00 & -1.2833E+00 \\ \hline
\end{tabular}}}{\caption{Reaction field energy of \ce{CN^-}. Radius on top row in Bohr and energies in Hartree}
\label{tab:Erdatacyan}}

  \ttabbox{
  \resizebox{\textwidth}{!}{
    \begin{tabular}{|l|r|r|r|r|r|r|r|r|r|r|r|r|r|r|r|}
  \hline
  basis & 3.6 & 3.7 & 3.8 & 3.9 & 4 & 4.1 & 4.2 & 4.3 & 4.4 & 4.5 & 4.6 & 4.7 & 4.8 & 4.9 & 5 \\ \hline
  Cc-pVDZ & -1.3712E-01 & -1.3341E-01 & -1.2990E-01 & -1.2657E-01 & -1.2340E-01 & -1.2039E-01 & -1.1753E-01 & -1.1479E-01 & -1.1219E-01 & -1.0969E-01 & -1.0731E-01 & -1.0502E-01 & -1.0284E-01 & -1.0074E-01 & -9.8723E-02 \\ \hline
  Cc-pVTZ & -1.3712E-01 & -1.3341E-01 & -1.2990E-01 & -1.2657E-01 & -1.2340E-01 & -1.2039E-01 & -1.1753E-01 & -1.1479E-01 & -1.1219E-01 & -1.0969E-01 & -1.0731E-01 & -1.0502E-01 & -1.0284E-01 & -1.0074E-01 & -9.8723E-02 \\ \hline
  Cc-pVQZ & -1.3712E-01 & -1.3341E-01 & -1.2990E-01 & -1.2657E-01 & -1.2340E-01 & -1.2039E-01 & -1.1753E-01 & -1.1479E-01 & -1.1219E-01 & -1.0969E-01 & -1.0731E-01 & -1.0502E-01 & -1.0284E-01 & -1.0074E-01 & -9.8723E-02 \\ \hline
  Cc-pV5Z & -1.3712E-01 & -1.3341E-01 & -1.2990E-01 & -1.2657E-01 & -1.2340E-01 & -1.2039E-01 & -1.1753E-01 & -1.1479E-01 & -1.1219E-01 & -1.0969E-01 & -1.0731E-01 & -1.0502E-01 & -1.0284E-01 & -1.0074E-01 & -9.8723E-02 \\ \hline
  Aug-cc-pVDZ & -1.3712E-01 & -1.3341E-01 & -1.2990E-01 & -1.2657E-01 & -1.2340E-01 & -1.2039E-01 & -1.1753E-01 & -1.1479E-01 & -1.1219E-01 & -1.0969E-01 & -1.0731E-01 & -1.0502E-01 & -1.0284E-01 & -1.0074E-01 & -9.8723E-02 \\ \hline
  Aug-cc-pVTZ & -1.3712E-01 & -1.3341E-01 & -1.2990E-01 & -1.2657E-01 & -1.2340E-01 & -1.2039E-01 & -1.1753E-01 & -1.1479E-01 & -1.1219E-01 & -1.0969E-01 & -1.0731E-01 & -1.0502E-01 & -1.0284E-01 & -1.0074E-01 & -9.8723E-02 \\ \hline
  Aug-cc-pVQZ & -1.3712E-01 & -1.3341E-01 & -1.2990E-01 & -1.2657E-01 & -1.2340E-01 & -1.2039E-01 & -1.1753E-01 & -1.1479E-01 & -1.1219E-01 & -1.0969E-01 & -1.0731E-01 & -1.0502E-01 & -1.0284E-01 & -1.0074E-01 & -9.8723E-02 \\ \hline
  Aug-cc-pV5Z & -1.3712E-01 & -1.3341E-01 & -1.2990E-01 & -1.2657E-01 & -1.2340E-01 & -1.2039E-01 & -1.1753E-01 & -1.1479E-01 & -1.1219E-01 & -1.0969E-01 & -1.0731E-01 & -1.0502E-01 & -1.0284E-01 & -1.0074E-01 & -9.8723E-02 \\ \hline
  daug-cc-pVDZ & -1.3712E-01 & -1.3341E-01 & -1.2990E-01 & -1.2657E-01 & -1.2340E-01 & -1.2039E-01 & -1.1753E-01 & -1.1479E-01 & -1.1219E-01 & -1.0969E-01 & -1.0731E-01 & -1.0502E-01 & -1.0284E-01 & -1.0074E-01 & -9.8723E-02 \\ \hline
  daug-cc-pVTZ & -1.3712E-01 & -1.3341E-01 & -1.2990E-01 & -1.2657E-01 & -1.2340E-01 & -1.2039E-01 & -1.1753E-01 & -1.1479E-01 & -1.1219E-01 & -1.0969E-01 & -1.0731E-01 & -1.0502E-01 & -1.0284E-01 & -1.0074E-01 & -9.8723E-02 \\ \hline
  daug-cc-pVQZ & -1.3712E-01 & -1.3341E-01 & -1.2990E-01 & -1.2657E-01 & -1.2340E-01 & -1.2039E-01 & -1.1753E-01 & -1.1479E-01 & -1.1219E-01 & -1.0969E-01 & -1.0731E-01 & -1.0502E-01 & -1.0284E-01 & -1.0074E-01 & -9.8723E-02 \\ \hline
  daug-cc-pV5Z & -1.3712E-01 & -1.3341E-01 & -1.2990E-01 & -1.2657E-01 & -1.2340E-01 & -1.2039E-01 & -1.1753E-01 & -1.1479E-01 & -1.1219E-01 & -1.0969E-01 & -1.0731E-01 & -1.0502E-01 & -1.0284E-01 & -1.0074E-01 & -9.8723E-02 \\ \hline
  mrchem & -1.4283E-01 & -1.3881E-01 & -1.3501E-01 & -1.3141E-01 & -1.2798E-01 & -1.2475E-01 & -1.2165E-01 & -1.1873E-01 & -1.1595E-01 & -1.1325E-01 & -1.1075E-01 & -1.0832E-01 & -1.0599E-01 & -1.0375E-01 & -1.0161E-01 \\ \hline
  variational & -1.3797E-01 & -1.3426E-01 & -1.3067E-01 & -1.2723E-01 & -1.2393E-01 & -1.2079E-01 & -1.1780E-01 & -1.1498E-01 & -1.1231E-01 & -1.0972E-01 & -1.0735E-01 & -1.0505E-01 & -1.0287E-01 & -1.0079E-01 & -9.8805E-02 \\ \hline
  \end{tabular}}}{\caption{Reaction field energy of \ce{Li^+}. Radius on top row in Bohr and energies in Hartree}
  \label{tab:Erdatalip}}
\end{sidewaystable}

\begin{table}[htbp]
\caption{Total energies for \ce{CH_3 CONH_2 } with \ac{ABC}, radii in Bohr and energies in Hartree}
\begin{tabular}{|l|r|r|r|}
\hline
basis & \multicolumn{1}{l|}{vacuum} & \multicolumn{1}{l|}{\ac{ABC}} & \multicolumn{1}{l|}{\ac{ABC} + 0.2} \\ \hline
Cc-pVDZ & -2.0800E+02 & -2.0801E+02 & -2.0801E+02 \\ \hline
Cc-pVTZ & -2.0806E+02 & -2.0807E+02 & -2.0807E+02 \\ \hline
Cc-pVQZ & -2.0808E+02 & -2.0809E+02 & -2.0809E+02 \\ \hline
Cc-pV5Z & -2.0808E+02 & -2.0809E+02 & -2.0809E+02 \\ \hline
Aug-cc-pVDZ & -2.0801E+02 & -2.0803E+02 & -2.0803E+02 \\ \hline
Aug-cc-pVTZ & -2.0806E+02 & -2.0808E+02 & -2.0808E+02 \\ \hline
Aug-cc-pVQZ & -2.0808E+02 & -2.0809E+02 & -2.0809E+02 \\ \hline
Aug-cc-pV5Z & -2.0808E+02 & -2.0810E+02 & -2.0809E+02 \\ \hline
daug-cc-pVDZ & -2.0801E+02 & -2.0803E+02 & -2.0803E+02 \\ \hline
daug-cc-pVTZ & -2.0806E+02 & -2.0808E+02 & -2.0808E+02 \\ \hline
daug-cc-pVQZ & -2.0808E+02 & -2.0809E+02 & -2.0809E+02 \\ \hline
daug-cc-pV5Z & -2.0808E+02 & \multicolumn{1}{l|}{} & \multicolumn{1}{l|}{} \\ \hline
mrchem & -2.0808E+02 & -2.0810E+02 & -2.0809E+02 \\ \hline
variational & -2.0808E+02 & -2.1077E+02 & -2.0984E+02 \\ \hline
\end{tabular}
\label{tab:acetamidrawdataabc}
\end{table}

\begin{table}[htbp]
\caption{Total energies for \ce{H_2O } with \ac{ABC}}
\begin{tabular}{|l|r|r|}
\hline
basis & \multicolumn{1}{l|}{R = vdw*1.2} & \multicolumn{1}{l|}{R + 0.2} \\ \hline
Cc-pVDZ & -7.6035E+01 & -7.6034E+01 \\ \hline
Cc-pVTZ & -7.6066E+01 & -7.6064E+01 \\ \hline
Cc-pVQZ & -7.6074E+01 & -7.6072E+01 \\ \hline
Cc-pV5Z & -7.6076E+01 & -7.6074E+01 \\ \hline
Aug-cc-pVDZ & -7.6050E+01 & -7.6049E+01 \\ \hline
Aug-cc-pVTZ & -7.6069E+01 & -7.6068E+01 \\ \hline
Aug-cc-pVQZ & -7.6075E+01 & -7.6073E+01 \\ \hline
Aug-cc-pV5Z & -7.6076E+01 & -7.6075E+01 \\ \hline
daug-cc-pVDZ & -7.6051E+01 & -7.6049E+01 \\ \hline
daug-cc-pVTZ & -7.6070E+01 & -7.6068E+01 \\ \hline
daug-cc-pVQZ & -7.6075E+01 & -7.6073E+01 \\ \hline
daug-cc-pV5Z & -7.6076E+01 & -7.6075E+01 \\ \hline
mrchem & -7.6078E+01 & -7.6076E+01 \\ \hline
variational & -7.6387E+01 & -7.6205E+01 \\ \hline
\end{tabular}
\label{tab:watrawdataabc}
\end{table}

\begin{table}[htbp]
\caption{Total energies for \ce{NO^+ } with \ac{ABC}}
\begin{tabular}{|l|r|r|}
\hline
basis & \multicolumn{1}{l|}{vdw} & \multicolumn{1}{l|}{R + 0.2} \\ \hline
Cc-pVDZ & -1.2906E+02 & -1.2905E+02 \\ \hline
Cc-pVTZ & -1.2910E+02 & -1.2909E+02 \\ \hline
Cc-pVQZ & -1.2911E+02 & -1.2910E+02 \\ \hline
Cc-pV5Z & -1.2911E+02 & -1.2910E+02 \\ \hline
Aug-cc-pVDZ & -1.2906E+02 & -1.2906E+02 \\ \hline
Aug-cc-pVTZ & -1.2910E+02 & -1.2909E+02 \\ \hline
Aug-cc-pVQZ & -1.2911E+02 & -1.2910E+02 \\ \hline
Aug-cc-pV5Z & -1.2911E+02 & -1.2910E+02 \\ \hline
daug-cc-pVDZ & -1.2906E+02 & -1.2906E+02 \\ \hline
daug-cc-pVTZ & -1.2910E+02 & -1.2909E+02 \\ \hline
daug-cc-pVQZ & -1.2911E+02 & -1.2910E+02 \\ \hline
daug-cc-pV5Z & -1.2911E+02 & -1.2910E+02 \\ \hline
mrchem & -1.2911E+02 & -1.2911E+02 \\ \hline
variational & -1.2916E+02 & -1.2913E+02 \\ \hline
\end{tabular}
\label{tab:noprawdataabc}
\end{table}


\begin{table}[htbp]
\caption{Total energies for \ce{CN^- } with \ac{ABC}}
\begin{tabular}{|l|r|r|}
\hline
basis & \multicolumn{1}{l|}{vdw} & \multicolumn{1}{l|}{R + 0.2} \\ \hline
Cc-pVDZ & -9.2414E+01 & -9.2410E+01 \\ \hline
Cc-pVTZ & -9.2448E+01 & -9.2444E+01 \\ \hline
Cc-pVQZ & -9.2457E+01 & -9.2453E+01 \\ \hline
Cc-pV5Z & -9.2460E+01 & -9.2456E+01 \\ \hline
Aug-cc-pVDZ & -9.2435E+01 & -9.2432E+01 \\ \hline
Aug-cc-pVTZ & -9.2454E+01 & -9.2450E+01 \\ \hline
Aug-cc-pVQZ & -9.2459E+01 & -9.2455E+01 \\ \hline
Aug-cc-pV5Z & -9.2460E+01 & -9.2457E+01 \\ \hline
daug-cc-pVDZ & -9.2435E+01 & -9.2432E+01 \\ \hline
daug-cc-pVTZ & -9.2454E+01 & -9.2450E+01 \\ \hline
daug-cc-pVQZ & -9.2459E+01 & -9.2455E+01 \\ \hline
daug-cc-pV5Z & -9.2460E+01 & -9.2457E+01 \\ \hline
mrchem & -9.2463E+01 & \multicolumn{1}{l|}{N/A} \\ \hline
variational & -9.5400E+01 & -9.4773E+01 \\ \hline
\end{tabular}
\label{tab:cyanrawdataabc}
\end{table}

\begin{table}[htbp]
\caption{Reaction field  energies for \ce{H_2O } with \ac{ABC}}
\begin{tabular}{|l|r|r|}
\hline
Cc-pVDZ & -8.8187E-03 & -7.2789E-03 \\ \hline
Cc-pVTZ & -9.0536E-03 & -7.4595E-03 \\ \hline
Cc-pVQZ & -9.0468E-03 & -7.4576E-03 \\ \hline
Cc-pV5Z & -9.0436E-03 & -7.4613E-03 \\ \hline
Aug-cc-pVDZ & -9.0449E-03 & -7.4627E-03 \\ \hline
Aug-cc-pVTZ & -8.9913E-03 & -7.4176E-03 \\ \hline
Aug-cc-pVQZ & -8.9649E-03 & -7.3946E-03 \\ \hline
Aug-cc-pV5Z & -8.9703E-03 & -7.3957E-03 \\ \hline
daug-cc-pVDZ & -9.0550E-03 & -7.4645E-03 \\ \hline
daug-cc-pVTZ & -8.9965E-03 & -7.4201E-03 \\ \hline
daug-cc-pVQZ & -8.9685E-03 & -7.3974E-03 \\ \hline
daug-cc-pV5Z & -8.9711E-03 & -7.3962E-03 \\ \hline
mrchem & -1.0386E-02 & -8.4765E-03 \\ \hline
variational & -3.1962E-01 & -1.3725E-01 \\ \hline
\end{tabular}
\label{tab:abcErwat}
\end{table}


\begin{table}[htbp]
\caption{Reaction field  energies for \ce{NO^+ } with \ac{ABC}}
\begin{tabular}{|l|r|r|}
\hline
basis & \multicolumn{1}{l|}{vdw} & \multicolumn{1}{l|}{Vdw +0.2} \\ \hline
Cc-pVDZ & -1.2929E-01 & -1.2239E-01 \\ \hline
Cc-pVTZ & -1.2916E-01 & -1.2227E-01 \\ \hline
Cc-pVQZ & -1.2909E-01 & -1.2223E-01 \\ \hline
Cc-pV5Z & -1.2900E-01 & -1.2218E-01 \\ \hline
Aug-cc-pVDZ & -1.2887E-01 & -1.2206E-01 \\ \hline
Aug-cc-pVTZ & -1.2902E-01 & -1.2216E-01 \\ \hline
Aug-cc-pVQZ & -1.2903E-01 & -1.2218E-01 \\ \hline
Aug-cc-pV5Z & -1.2900E-01 & -1.2217E-01 \\ \hline
daug-cc-pVDZ & -1.2889E-01 & -1.2208E-01 \\ \hline
daug-cc-pVTZ & -1.2900E-01 & -1.2216E-01 \\ \hline
daug-cc-pVQZ & -1.2902E-01 & -1.2218E-01 \\ \hline
daug-cc-pV5Z & -1.2900E-01 & -1.2217E-01 \\ \hline
mrchem & -1.3411E-01 & -1.2670E-01 \\ \hline
variational & -1.8326E-01 & -1.5337E-01 \\ \hline
\end{tabular}
\label{tab:abcErnop}
\end{table}

\begin{table}[htbp]
\caption{Reaction field  energies for \ce{CN^- } with \ac{ABC}}
\begin{tabular}{|l|r|r|}
\hline
basis & \multicolumn{1}{l|}{vdw} & \multicolumn{1}{l|}{Vdw + 0.2} \\ \hline
Cc-pVDZ & -1.6388E+01 & -1.6383E+01 \\ \hline
Cc-pVTZ & -1.6391E+01 & -1.6387E+01 \\ \hline
Cc-pVQZ & -1.6393E+01 & -1.6389E+01 \\ \hline
Cc-pV5Z & -1.6393E+01 & -1.6389E+01 \\ \hline
Aug-cc-pVDZ & -1.6394E+01 & -1.6390E+01 \\ \hline
Aug-cc-pVTZ & -1.6393E+01 & -1.6390E+01 \\ \hline
Aug-cc-pVQZ & -1.6393E+01 & -1.6390E+01 \\ \hline
Aug-cc-pV5Z & -1.6393E+01 & -1.6389E+01 \\ \hline
daug-cc-pVDZ & -1.6394E+01 & -1.6390E+01 \\ \hline
daug-cc-pVTZ & -1.6393E+01 & -1.6390E+01 \\ \hline
daug-cc-pVQZ & -1.6393E+01 & -1.6390E+01 \\ \hline
daug-cc-pV5Z & -1.6393E+01 & -1.6389E+01 \\ \hline
mrchem & -1.1376E-01 & \multicolumn{1}{l|}{N/A} \\ \hline
variational & -3.0512E+00 & -2.4239E+00 \\ \hline
\end{tabular}
\label{tab:abcErcyan}
\end{table}

\begin{table}[htbp]
\caption{Reaction field  energies for \ce{CH_3 CONH_2 } with \ac{ABC}}
\begin{tabular}{|l|r|r|}
\hline
basis & \multicolumn{1}{l|}{vdw} & \multicolumn{1}{l|}{Vdw+0.2} \\ \hline
Cc-pVDZ & -1.4239E-02 & -1.1986E-02 \\ \hline
Cc-pVTZ & -1.5055E-02 & -1.2700E-02 \\ \hline
Cc-pVQZ & -1.5321E-02 & -1.2958E-02 \\ \hline
Cc-pV5Z & -1.5374E-02 & -1.3011E-02 \\ \hline
Aug-cc-pVDZ & -1.5502E-02 & -1.3105E-02 \\ \hline
Aug-cc-pVTZ & -1.5390E-02 & -1.3029E-02 \\ \hline
Aug-cc-pVQZ & -1.5352E-02 & -1.2996E-02 \\ \hline
Aug-cc-pV5Z & -1.5354E-02 & -1.2993E-02 \\ \hline
daug-cc-pVDZ & -1.5446E-02 & -1.3050E-02 \\ \hline
daug-cc-pVTZ & -1.5374E-02 & -1.3012E-02 \\ \hline
daug-cc-pVQZ & -1.5353E-02 & -1.2996E-02 \\ \hline
daug-cc-pV5Z & \multicolumn{1}{l|}{N/A} & \multicolumn{1}{l|}{N/A} \\ \hline
mrchem & -1.7496E-02 & -1.4615E-02 \\ \hline
variational & -2.6924E+00 & -1.7618E+00 \\ \hline
\end{tabular}
\label{tab:abcEracetamid}
\end{table}


\begin{acronym}
\acro{AUS}[\href{https://www.sigma2.no/content/advanced-user-support}{AUS}]{Numerical Methods in Quantum Chemistry}
\acro{BO}{Born-Oppenheimer}
\acro{CTCC}[\href{http://www.ctcc.no}{CTCC}]{Centre for Theoretical and Computational Chemistry}
\acro{DC}{Dielectric Continuum}
\acro{DFT}{Density Functional Theory}
\acro{EFP}{Effective Fragment Potential}
\acro{EU}{European Union}
\acro{HF}{Hartree-Fock}
\acro{Hylleraas}[\href{https://www.mn.uio.no/hylleraas/english/}{Hylleraas}]{Hylleraas
  Centre for Quantum Molecular Sciences}
\acro{HPC}{High Performance Computing}
\acro{KTH}{Royal Institute of Technology}
\acro{LDA}{Local Density Approximation}
\acro{MCD}{Magnetic Circular Dichroism}
\acro{MCSCF}{Multiconfiguration Self Consistent Field}
\acro{MM}{Molecular Mechanics}
\acro{MW}{Multiwavelet}
\acro{NFR}{Norwegian Research Council}
\acro{NMQC}[\href{http://www.ctcc.no/events/conferences/2015/numeric-conference/}{NMQC}]{Numerical Methods in Quantum Chemistry}
\acro{NOTUR}[\href{https://www.notur.no/}{NOTUR}]{Norwegian Metacenter for Computational Science}
\acro{PCM}{Polarizable Continuum Model}
\acro{PI}{Primcipal Investigator}
\acro{QC}{Quantum Chemistry}
\acro{QM}{Quantum Mechanics}
\acro{QM/MM}{Quantum Mechanics/Molecular Mechanics}
\acro{ROA}{Raman Optical Activity}
\acro{SC}{semiconductor}
\acro{SCF}{Self Consistent Field}
\acro{SHG}{Second Harmonic Genertation}
\acro{STSM}{Short-term scientific mission}
\acro{TPA}{Two-Photon Absorption}
\acro{WP}{Work Package}
\acro{CBS}{Complete Basis Set}
\acro{TCG}{Theoretical Chemistry Group}
\acro{vdW}{van der Waals}
\acro{SE}{Schrödinger Equation}
\acro{PES}{Potential Energy Surface}
\acro{LCAO}{Linear Combination of Atomic Orbitals}
\acro{MRA}{Multi-Resolution Analysis}
\acro{NS}{Nonstandard}
\acro{GPE}{Generalized Poisson Equation}
\acro{COSMO}{Conductor-like Screening model}
\acro{IEF}{Integral equation formalism}
\acro{SVPE}{Surface and Volume Polarization for Electrostatic}
\acro{ASC}{Apparent Surface Charge}
\acro{SCRF}{Self Consistent Reaction Field}
\acro{STO}{Slater-type orbitals}
\end{acronym}

\biblio
\end{document}
