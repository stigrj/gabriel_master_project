\makeatletter
\def\input@path{{../}}
\makeatother
\documentclass[../master_thesis.tex]{subfiles}
\begin{document}
\chapter{Quantum Chemistry}
\section{Eigenvalue Problems}
Consider a generic system (the system one is most occuppied with in theoretical
chemistry is a molecule or an atom). The system can be described by a
wavefunction $\Psi$. This wavefunction, by definition, has all information
about the system, which can be extracted by solving the Eigenvalue equation
with a characteristic operator $\vartheta$ for an observable property $ e $ to
be extracted as an eigenvalue of the operator applied on the wavefunction
\cite{Cramer:2004}:
\begin{equation}
  \vartheta\Psi = e\Psi\label{eq:eigenequation}
\end{equation}

The wavefunction is usually normalized, meaning that the integral in Equation
\ref{eq:normwf} applies. The integral below can be written in bra-ket
\cite{Jensen:2017} notation, which is shown in Equation \ref{eq:normwf}.
\begin{equation}
  \braket{\Psi|\Psi} = \int\limits_{-\infty}^{\infty}
  \Psi^*(\vec{r})\Psi(\vec{r})\mathrm{\,d}\vec{r} = 1\label{eq:normwf}
\end{equation} 
Where $\Psi^*$ is the complex conjugate wavefunction and $\vec{r}$ is a
coordinate vector in which the wavefunction is evaluated over all space.
Bra-ket notation is what will be used from now on to write integrals in a more
compact manner.

The Equation \ref{eq:normwf} calculates the probability of finding the system
that the wavefunction $\Psi$ describes in all of space. The expectation value
of an arbitrary operator $\vartheta$ with eigenvalue $e$ is calculated as
follows \cite{Atkins:2014}.
\begin{equation}
  \begin{aligned}
  \braket{\Psi|\vartheta|\Psi} &= \braket{\Psi|e|\Psi} \\
   \braket{\Psi|\vartheta|\Psi}&= e\braket{\Psi|\Psi} \\
   \frac{\braket{\Psi|\vartheta|\Psi}}{\braket{\Psi|\Psi}}&= e\label{eq:eval}
  \end{aligned}
\end{equation}
We can apply Equation \ref{eq:normwf} on the denominator on the l.h.s. of the
last step on the Equation \ref{eq:eval} is one in a normalized wavefunction.

From Equation \ref{eq:eigenequation} we can see that applying the operator on
the wavefunction yields the eigenvalue of said operator. Given that the
eigenvalue $e$ is a constant it can be taken out of the integral. The integral
then yields the eigenvalue, which is also the expectation value of said
operator.

In the case that the Wavefunction $\Psi$ is not an eigenfunction of the
operator, the wavefunction can be constructed by a linear combination of
orthogonal and normalized (orthonormal\cite{Jensen:2017}) eigenfunctions
$\Psi_i$ of the operator \cite{Atkins:2014} scaled by scalars $c_i$:
\begin{equation}
  \Psi = c_1\Psi_1 + c_2\Psi_2 + ... + c_n\Psi_n
\end{equation}

The fact that the eigenfunctions are orthogonal means that:
\begin{equation}
  \braket{\Psi_i|\Psi_j} = \delta _{ij}
\end{equation}
Where $ \delta_{ij} $ is one when $i=j$ and zero otherwise \cite{Cramer:2004}.

Finding the expectation value for each eigen of the operator can then be done
as a sum of the expectation values $e_i$ of each eigenfunction:
\begin{equation}
  \begin{align}
    \braket{\Psi|\vartheta|\Psi} &=
    \sum\limits_i^n\braket{c_i\Psi_i|\vartheta|c_i\Psi_i} \\
    &= \sum\limits_i^n\braket{c_i\Psi_i|e_i|c_i\Psi_i} \\
    &= \sum\limits_i^nc_i^2e_i\braket{\Psi_i|\Psi_i} \\
    \braket{\Psi|\vartheta|\Psi} &= \sum\limits_i^nc_i^2e_i
  \end{align}
\end{equation}
With this knowledge we can state that the expectation value of an operator
applied on any wavefunction can always be calculated as a sum of weighted
eigenvalues for sums of orthonormal eigenfunctions.

Furthermore, one can see that the probaiblity of the wavefunction is equal to
$\sum_i^nc_i^2$. We know that in a normalized wavefunction the
probability is equals to one, so we just need constants that satisfy the
following equation.
\begin{equation}
  \braket{\Psi|\Psi} = 1 = \sum\limits_i^nc_i^2
\end{equation}


\section{The \ac{SE}}
%show the time dependent SE and derive the time independent SE
The information that is of most interest in computational chemistry is the
total energy of the system $ E $. The operator for which the energy $E$ of the
system is its Expectation value is called the Hamiltonian $H$. Plugging this
into \ref{eq:eigenequation} yields the time independent \ac{SE}
\cite{Cramer:2004, Jensen:2017}:
\begin{equation}
  H\Psi = E\Psi\label{eq:SE}
\end{equation}
From Equation \ref{eq:eval} one can see that the solution of Equation
\ref{eq:SE} is the expectation value of the Hamiltonian.



\section{Two particle system}
Consider a one electron system (such as \ce{H} or \ce{He+}) which contains an
electron and a nucleus, which are described using three dimensional coordinates
$ x, y $ and $ z $. The Hamiltonian contains terms for the kinetic $T$ and
potential energy $V$ contributions of both the nucleus $ N $ and the electron
$ e $. In a one electron system in vaccuum the Hamiltonian would be as follows
\cite{Jensen:2017, Cramer:2004}.
\begin{equation}
  \begin{aligned}
    H   &= T_N + T_e + V \\
    T_N &= -\frac{\hbar^2}{2m_N}\nabla^2_N \\
    T_e &= -\frac{\hbar^2}{2m_e}\nabla^2_e \\
    V   &= -\frac{Z}{||\vec{r}_N - \vec{r}_e||} \label{eq:twopH}
  \end{aligned}
\end{equation}
Where the operator $ \nabla^2$ is the Laplacian computing the second partial
derivative $ \frac{\delta^2}{\delta x^2} + \frac{\delta^2}{\delta y^2} +
\frac{\delta^2}{\delta z^2} $ for the coordinates of the electron $\vec{r}_e$
and the nucleus $\vec{r}_N$, $ m $ is the mass of the electron and the nucleus
depending on the subscript used. The potential is the coulomb interaction
between the electrons, which is calculated by dividing the atomic number of the
nucleus $ Z $ with the distance between the electron and the nucleus
$ ||\vec{r}_N - \vec{r}_e|| $.

We can see that the kinetic energy operators are separable, as they are only
dependent on their respective particlecoordinates. The potential energy
operator, on the other hand, is not separable, as it is dependent in the
coordinates of both the nucleus and the electron.

In order to solve the the \ac{SE} analytically one must first change into a
center of mass coordinate system :
\begin{equation}
  \begin{alignat}{3}
    X &= \frac{m_Nx_N + m_ex_e}{m_N + m_e} ~&;~ x &= x_N - x_e \\
    Y &= \frac{m_Ny_N + m_ey_e}{m_N + m_e} ~&;~ y &= y_N - y_e \\
    Z &= \frac{m_Nz_N + m_ez_e}{m_N + m_e} ~&;~ z &=
    z_N - z_e \label{eq:cmparam}
  \end{alignat}
\end{equation}
Where the $XYZ$-coordinates define a system centered in the center of mass,
while the $xyz$-coordinates specify the relative position of the two particles
\cite{Jensen:2017}. Applying this coordinate system to the Hamiltonian in
Equation \ref{eq:twopH} yields:
\begin{equation}
  H = -\frac{1}{2}\nabla^2_{XYZ} -\frac{1}{2\mu}\nabla^2_{xyz} -
  \frac{Z}{\sqrt{x^2 + y^2 + z^2}}\label{eq:twopHcm}
\end{equation}
Where the first term in Equation \ref{eq:twopHcm} describes the motion of the
whole system with respect to a fixed coordinate system. The second term
describes the relative motion of a pseudo-particle with reduced mass:
\begin{equation}
  \mu = \frac{m_Nm_e}{m_N + m_e} = \frac{m_e}{1 + \frac{m_e}{m_N}}
\end{equation}
Given that the nucleus is more massive than the electron (the nucleus is $1800$
times more massive than the electron in a Hydrogen atom \cite{Jensen:2017})
$\mu$ can be approximated to the mass of the electron $m_e$. This lets us
assume that the nucleus is stationary in relation with the electron.

The third term is the potential energy which still is dependent on the distance
between the two particles. In this center of mass system the distance is just
determined by the $xyz$-coordinates.

The motion of the particles occurs in three dimensions, therefore it is
advantageous to use polar coordinates to solve the \ac{SE}. In this coordinate
system $r$ describes the distance to the center, $\theta$ describing the angle
of the particle with respect to the $z$-axis and $\varphi$ describes the polar
angle on the $xy$-plane. Applying this to the Hamiltonian in Equation
\ref{eq:twopHcm} gives:
\begin{equation}
  \begin{align}
    H &= -\frac{1}{2\mu}\nabla^2_{r\theta\varphi} - \frac{Z}{r}\\
    \nabla^2_{r\theta\varphi} &=
    \frac{1}{r^2}\frac{\delta}{\delta r}r^2\frac{\delta}{\delta r}
    + \frac{1}{r^2 \sin{\theta}}\frac{\delta}{\delta\theta}\sin{\theta}
      \frac{\delta}{\delta\theta}
    + \frac{1}{r^2\sin^2{\theta}}\frac{\delta^2}{\delta\varphi^2}
    \label{eq:SphHcm}
  \end{align}
\end{equation}
Which can be solved using different differential equation methods. The solution
is parametrized to the three quantum number n, l and m corresponding to the
spherical coordinates. Solving the resulting differential equation with the
Hamiltonian in Equation \ref{eq:SphHcm} and the wavefunction gives a series of
orthonormal solutions to the wavefunction called orbitals \cite{Jensen:2017}.
%more about the properties of these orbitals
\section{Many body systems}

For bigger systems, there is no practical way to analytically solve the
\ac{SE}. For one, for each particle, the amount of dimensions that need to be
evaluated increases by three, that is, one can expect the wavefunction
dimension to increase by a factor of $3N$ for each particle $N$
\cite{Cramer:2004}.

Additionally, the potential energy operator becomes more complicated,
as it no would not just have the attractive forces between electron-nucleus,
but also the repulsive forces between all the electrons. Both of these problems
add more terms per particle and thus would be impossible to be solved in a
realistic timeframe. %cite
%don't know what to write here

\subsection{Variational Principle}

\subsection{\ac{BO} approximation}

\subsection{Orbitals}








\begin{acronym}
\acro{AUS}[\href{https://www.sigma2.no/content/advanced-user-support}{AUS}]{Numerical Methods in Quantum Chemistry}
\acro{BO}{Born-Oppenheimer}
\acro{CTCC}[\href{http://www.ctcc.no}{CTCC}]{Centre for Theoretical and Computational Chemistry}
\acro{DC}{Dielectric Continuum}
\acro{DFT}{Density Functional Theory}
\acro{EFP}{Effective Fragment Potential}
\acro{EU}{European Union}
\acro{HF}{Hartree-Fock}
\acro{Hylleraas}[\href{https://www.mn.uio.no/hylleraas/english/}{Hylleraas}]{Hylleraas
  Centre for Quantum Molecular Sciences}
\acro{HPC}{High Performance Computing}
\acro{KTH}{Royal Institute of Technology}
\acro{LDA}{Local Density Approximation}
\acro{MCD}{Magnetic Circular Dichroism}
\acro{MCSCF}{Multiconfiguration Self Consistent Field}
\acro{MM}{Molecular Mechanics}
\acro{MW}{Multiwavelet}
\acro{NFR}{Norwegian Research Council}
\acro{NMQC}[\href{http://www.ctcc.no/events/conferences/2015/numeric-conference/}{NMQC}]{Numerical Methods in Quantum Chemistry}
\acro{NOTUR}[\href{https://www.notur.no/}{NOTUR}]{Norwegian Metacenter for Computational Science}
\acro{PCM}{Polarizable Continuum Model}
\acro{PI}{Primcipal Investigator}
\acro{QC}{Quantum Chemistry}
\acro{QM}{Quantum Mechanics}
\acro{QM/MM}{Quantum Mechanics/Molecular Mechanics}
\acro{ROA}{Raman Optical Activity}
\acro{SC}{semiconductor}
\acro{SCF}{Self Consistent Field}
\acro{SHG}{Second Harmonic Genertation}
\acro{STSM}{Short-term scientific mission}
\acro{TPA}{Two-Photon Absorption}
\acro{WP}{Work Package}
\acro{CBS}{Complete Basis Set}
\acro{TCG}{Theoretical Chemistry Group}
\acro{vdW}{van der Waals}
\acro{SE}{Schrödinger Equation}
\acro{PES}{Potential Energy Surface}
\acro{LCAO}{Linear Combination of Atomic Orbitals}
\acro{MRA}{Multi-Resolution Analysis}
\acro{NS}{Nonstandard}
\acro{GPE}{Generalized Poisson Equation}
\acro{COSMO}{Conductor-like Screening model}
\acro{IEF}{Integral equation formalism}
\acro{SVPE}{Surface and Volume Polarization for Electrostatic}
\acro{ASC}{Apparent Surface Charge}
\acro{SCRF}{Self Consistent Reaction Field}
\acro{STO}{Slater-type orbitals}
\end{acronym}

\biblio
\end{document}
