\makeatletter
\def\input@path{{../}}
\makeatother
\documentclass[../master_thesis.tex]{subfiles}
\begin{document}
\chapter{Computational Chemistry}
\section{Theoretical Chemistry}
\subsection{Time Independent \ac{SE}}

Consider a generic system (the system one is most occuppied with in theoretical
chemistry is a molecule or an atom). The system can be described by a
wavefunction $\Psi$. This wavefunction, by definition, has all information
about the system, which can be extracted by solving the Eigenvalue equation
with a characteristic operator $\vartheta$ for an observable property $ e $ to
 be extracted \cite{Cramer:2004}:
\begin{equation}
  \vartheta\Psi = e\Psi\label{eq:eigenequation}
\end{equation}

The information that is of most interest in computational chemistry is the
total energy of the system $ E $. The operator for the energy of the system is
called the Hamiltonian $H$. Plugging this into \ref{eq:eigenequation} yields
the time independent \ac{SE} \cite{Cramer:2004, Jensen:2017}:
\begin{equation}
  H\Psi = E\Psi\label{eq:SE}
\end{equation}

Consider a one electron system (such as \ce{H} or \ce{He+}) which contains an
electron and a nucleus, which are described using three dimensional coordinates
$ x, y $ and $ z $. The Hamiltonian contains terms for the kinetic $T$ and
potential energy $V$ contributions of both the nucleus $ N $ and the electron
$ e $. In a one electron system in vaccuum the Hamiltonian would be as follows
\cite{Jensen:2017, Cramer:2004}.
\begin{equation}
  \begin{aligned}
    H   &= T_N + T_e + V \\
    T_N &= -\frac{\hbar^2}{2m_N}\nabla^2_N \\
    T_e &= -\frac{\hbar^2}{2m_e}\nabla^2_e \\
    V   &= -\frac{Z}{||\vec{r}_N - \vec{r}_e||} \label{eq:twopH}
  \end{aligned}
\end{equation}
Where the operator $ \nabla^2$ is the Laplacian computing the second partial
derivative $ \frac{\delta^2}{\delta x^2} + \frac{\delta^2}{\delta y^2} +
\frac{\delta^2}{\delta z^2} $ for the coordinates of the electron $\vec{r}_e$
and the nucleus $\vec{r}_N$, $ m $ is the mass of the electron and the nucleus
depending on the subscript used. The potential is the coulomb interaction
between the electrons, which is calculated by dividing the atomic number of the
nucleus $ Z $ with the distance between the electron and the nucleus
$ ||\vec{r}_N - \vec{r}_e|| $.

We can see that the kinetic energy operators are separable, as they are only
dependent on their respective particlecoordinates. The potential energy
operator, on the other hand, is not separable, as it is dependent in the
coordinates of both the nucleus and the electron.

In order to solve the the \ac{SE} analytically one must first change into a
center of mass coordinate system :
\begin{equation}
  \begin{alignat}{3}
    X &= \frac{m_Nx_N + m_ex_e}{m_N + m_e} ~&;~ x &= x_N - x_e \\
    Y &= \frac{m_Ny_N + m_ey_e}{m_N + m_e} ~&;~ y &= y_N - y_e \\
    Z &= \frac{m_Nz_N + m_ez_e}{m_N + m_e} ~&;~ z &=
    z_N - z_e \label{eq:cmparam}
  \end{alignat} %no problem here
\end{equation}
Where the $XYZ$-coordinates define a system centered in the center of mass,
while the $xyz$-coordinates specify the relative position of the two particles
\cite{Jensen:2017}. Applying this coordinate system to the Hamiltonian in
Equation \ref{eq:twopH} yields:
\begin{equation}
  H = -\frac{1}{2}\nabla^2_{XYZ} -\frac{1}{2\mu}\nabla^2_{xyz} -
  \frac{Z}{\sqrt{x^2 + y^2 + z^2}}\label{eq:twopHcm}
\end{equation}
Where the first term in Equation \ref{eq:twopHcm} describes the motion of the
whole system with respect to a fixed coordinate system. The second term
describes the relative motion of a pseudo-particle with reduced mass:
\begin{equation}
  \mu = \frac{m_Nm_e}{m_N + m_e} = \frac{m_e}{1 + \frac{m_e}{m_N}}
\end{equation}
Given that the nucleus is more massive than the electron (the nucleus is $1800$
times more massive than the electron in a Hydrogen atom \cite{Jensen:2017})
$\mu$ can be approximated to the mass of the electron $m_e$. This lets us
assume that the nucleus is stationary in relation with the electron.

The third term is the potential energy which still is dependent on the distance
between the two particles. In this center of mass system the distance is just
determined by the $xyz$-coordinates.

The motion of the particles occurs in three dimensions, therefore it is
advantageous to use polar coordinates to solve the \ac{SE}. In this coordinate
system $r$ describes the distance to the center, $\theta$ describing the angle
of the particle with respect to the $z$-axis and $\varphi$ describes the polar
angle on the $xy$-plane. Applying this to the Hamiltonian in Equation
\ref{eq:twopHcm} gives:
\begin{equation}
  \begin{align}
    H &= -\frac{1}{2\mu}\nabla^2_{r\theta\varphi} - \frac{Z}{r}\\
    \nabla^2_{r\theta\varphi} &=
    \frac{1}{r^2}\frac{\delta}{\delta r}r^2\frac{\delta}{\delta r}
    + \frac{1}{r^2 \sin{\theta}}\frac{\delta}{\delta\theta}\sin{\theta}
      \frac{\delta}{\delta\theta}
    + \frac{1}{r^2\sin{\theta}^2}\frac{\delta^2}{\delta\varphi^2}
    \label{eq:SphHcm}
  \end{align}
\end{equation}
Which can be solved using different differential equation methods. The solution
is parametrized to the three quantum number n, l and m corresponding to the
spherical coordinates. Solving the \ac{SE} with the Hamiltonian in Equation
\ref{eq:SphHcm} gives a series of orthonormal solutions to the wavefunction
called orbitals.



\begin{acronym}
\acro{AUS}[\href{https://www.sigma2.no/content/advanced-user-support}{AUS}]{Numerical Methods in Quantum Chemistry}
\acro{BO}{Born-Oppenheimer}
\acro{CTCC}[\href{http://www.ctcc.no}{CTCC}]{Centre for Theoretical and Computational Chemistry}
\acro{DC}{Dielectric Continuum}
\acro{DFT}{Density Functional Theory}
\acro{EFP}{Effective Fragment Potential}
\acro{EU}{European Union}
\acro{HF}{Hartree-Fock}
\acro{Hylleraas}[\href{https://www.mn.uio.no/hylleraas/english/}{Hylleraas}]{Hylleraas
  Centre for Quantum Molecular Sciences}
\acro{HPC}{High Performance Computing}
\acro{KTH}{Royal Institute of Technology}
\acro{LDA}{Local Density Approximation}
\acro{MCD}{Magnetic Circular Dichroism}
\acro{MCSCF}{Multiconfiguration Self Consistent Field}
\acro{MM}{Molecular Mechanics}
\acro{MW}{Multiwavelet}
\acro{NFR}{Norwegian Research Council}
\acro{NMQC}[\href{http://www.ctcc.no/events/conferences/2015/numeric-conference/}{NMQC}]{Numerical Methods in Quantum Chemistry}
\acro{NOTUR}[\href{https://www.notur.no/}{NOTUR}]{Norwegian Metacenter for Computational Science}
\acro{PCM}{Polarizable Continuum Model}
\acro{PI}{Primcipal Investigator}
\acro{QC}{Quantum Chemistry}
\acro{QM}{Quantum Mechanics}
\acro{QM/MM}{Quantum Mechanics/Molecular Mechanics}
\acro{ROA}{Raman Optical Activity}
\acro{SC}{semiconductor}
\acro{SCF}{Self Consistent Field}
\acro{SHG}{Second Harmonic Genertation}
\acro{STSM}{Short-term scientific mission}
\acro{TPA}{Two-Photon Absorption}
\acro{WP}{Work Package}
\acro{CBS}{Complete Basis Set}
\acro{TCG}{Theoretical Chemistry Group}
\acro{vdW}{van der Waals}
\acro{SE}{Schrödinger Equation}
\acro{PES}{Potential Energy Surface}
\acro{LCAO}{Linear Combination of Atomic Orbitals}
\acro{MRA}{Multi-Resolution Analysis}
\acro{NS}{Nonstandard}
\acro{GPE}{Generalized Poisson Equation}
\acro{COSMO}{Conductor-like Screening model}
\acro{IEF}{Integral equation formalism}
\acro{SVPE}{Surface and Volume Polarization for Electrostatic}
\acro{ASC}{Apparent Surface Charge}
\acro{SCRF}{Self Consistent Reaction Field}
\acro{STO}{Slater-type orbitals}
\end{acronym}

\biblio
\end{document}
